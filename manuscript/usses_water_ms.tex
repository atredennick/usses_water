%% Submissions for peer-review must enable line-numbering
%% using the lineno option in the \documentclass command.
%%
%% Preprints and camera-ready submissions do not need
%% line numbers, and should have this option removed.
%%
%% Please note that the line numbering option requires
%% version 1.1 or newer of the wlpeerj.cls file, and
%% the corresponding author info requires v1.2

\documentclass[fleqn,10pt,lineno]{wlpeerj} % for journal submissions

% ZNK -- Adding headers for pandoc

\setlength{\emergencystretch}{3em}
\providecommand{\tightlist}{
\setlength{\itemsep}{0pt}\setlength{\parskip}{0pt}}
\usepackage{lipsum}
\usepackage[unicode=true]{hyperref}
\usepackage{longtable}



\usepackage{setspace} \usepackage{todonotes} \usepackage{rotating}
\usepackage{color, soul} \usepackage{sectsty} \usepackage{bm,mathrsfs}
\usepackage{lipsum}

\title{Ecosystem functional response across precipitation extremes in a
sagebrush steppe}

\author[1,*]{Andrew T. Tredennick}

\corrauthor[1,*]{Andrew T. Tredennick}{\href{mailto:atredenn@gmail.com}{\nolinkurl{atredenn@gmail.com}}}
\author[1,2,*]{Andrew R. Kleinhesselink}

\author[3]{J. Bret Taylor}

\author[1]{Peter B. Adler}


\affil[1]{Department of Wildland Resources and the Ecology Center, Utah State
University, Logan, Utah 84322}
\affil[2]{Department of Ecology and Evolutionary Biology, University of
California, Los Angeles, Los Angeles, California 90095}
\affil[3]{United States Department of Agriculture, Agricultural Research Service,
U.S. Sheep Experiment Station, Dubois, Idaho 83423}


%
% \author[1]{First Author}
% \author[2]{Second Author}
% \affil[1]{Address of first author}
% \affil[2]{Address of second author}
% \corrauthor[1]{First Author}{f.author@email.com}

% 

\begin{abstract}
\textbf{Background.} Precipitation is predicted to become more variable
in the western United States, meaning years of above and below average
precipitation will become more common. Periods of extreme precipitation
are major drivers of interannual variability in ecosystem functioning in
water limited communities, but how ecosystems respond to these extremes
over the long-term may shift with precipitation means and variances.
Long-term changes in ecosystem functional response could reflect
compensatory changes in species composition or species reaching
physiological thresholds at extreme precipitation levels.

\textbf{Methods.} We conducted a five year precipitation manipulation
experiment in a sagebrush steppe ecosystem in Idaho, United States. We
used drought and irrigation treatments (approximately 50\%
decrease/increase) to investigate whether ecosystem functional response
remains consistent under sustained high or low precipitation. We
recorded data on aboveground net primary productivity (ANPP), species
abundance, and soil moisture. We fit a generalized linear mixed effects
model to determine if the relationship between ANPP and soil moisture
differed among treatments. We used nonmetric multidimensional scaling to
quantify community composition over the five years.

\textbf{Results.} Ecosystem functional response, defined as the
relationship between soil moisture and ANPP, was similar among
irrigation and control treatments, but the drought treatment had a
greater slope than the control treatment. However, all estimates for the
effect of soil moisture on ANPP overlapped zero, indicating the
relationship is weak and uncertain regardless of treatment. There was
also large spatial variation in ANPP within-years, which contributes to
the uncertainty of the soil moisture effect. Plant community composition
was remarkably stable over the course of the experiment and did not
differ among treatments.

\textbf{Discussion.} Despite some evidence that ecosystem functional
response became more sensitive under sustained drought conditions, the
response of ANPP to soil moisture was consistently weak and community
composition was stable. The similarity of ecosystem functional responses
across treatments was not related to compensatory shifts at the plant
community level, but instead may reflect the insensitivity of the
dominant species to soil moisture. These species may be successful
precisely because they have evolved life history strategies which buffer
them against precipitation variability.
% Dummy abstract text. Dummy abstract text. Dummy abstract text. Dummy abstract text. Dummy abstract text. Dummy abstract text. Dummy abstract text. Dummy abstract text. Dummy abstract text. Dummy abstract text. Dummy abstract text.
\end{abstract}

\begin{document}

\flushbottom
\maketitle
\thispagestyle{empty}

\definecolor{blue}{rgb}{0,0,0.7}\newcommand{\new}{\textcolor{blue}}

\newcommand\blfootnote[1]{%
  \begingroup
  \renewcommand\thefootnote{}\footnote{#1}%
  \addtocounter{footnote}{-1}%
  \endgroup
}\blfootnote{*Authors contributed equally to this work.}

\reversemarginpar

\hypertarget{introduction}{%
\section{INTRODUCTION}\label{introduction}}

\new{Water availability is a major driver of annual net primary productivity (ANPP) in grassland ecosystems}
(Huxman et al. 2004; Hsu, Powell, and Adler 2012).
\new{Therefore, projected changes in precipitation regimes, associated with global climate change, will impact grassland ecosystem functioning.}
At any given site, the relationship between ANPP and water availability
(e.g., soil moisture) can be characterized by regressing historical
observations of ANPP on observations of soil moisture. Fitted functional
responses can then be used to infer how ANPP may change under future
precipitation regimes (e.g., Hsu, Powell, and Adler 2012). A problem
with this approach is that it requires extrapolation if future
precipitation falls outside the historical range of variability (Smith
2011; Peters et al. 2012). For example, the soil moisture-ANPP
relationship may be linear within the historical range of interannual
variation, but could saturate at higher levels of soil moisture.
Saturating relationships are actually common (Hsu, Powell, and Adler
2012; Laureano A Gherardi and Sala 2015), perhaps because other
resources, like nitrogen, become more limiting in wet years than dry
years. Failure to accurately estimate the curvature of the soil
moisture-ANPP relationship will lead to over- or underprediction of ANPP
under extreme precipitation (Peters et al. 2012).

Another problem with relying on historical ecosystem functional
responses to predict impacts of altered precipitation regimes is that
these relationships themselves might shift over the long-term. Shifts in
species identities and/or abundances can alter an ecosystem's functional
response to water availability because different species have different
physiological thresholds. Smith, Knapp, and Collins (2009) introduced
the `Hierarchical Response Framework' for understanding the interplay of
community composition and ecosystem functioning in response to long-term
shifts in resources. In the near term, ecosystem functioning such as
ANPP will reflect the physiological responses of individual species to
the manipulated resource level. For example, ANPP may initially decline
under simulated drought because the community consisted of
drought-intolerant species (Hoover, Knapp, and Smith 2014), but
functioning may recover over longer time spans as new species colonize
or resident species reorder in relative abundance. It is also possible
that ecosystem functioning eventually shifts to a new mean state,
reflecting the suite of species in the new community (Knapp, Briggs, and
Smith 2012).

Experimental manipulations of limiting resources, like precipitation,
offer a way to test how ecosystems will respond to resource levels that
fall outside the historical range of variability (Avolio et al. 2015;
Laureano A. Gherardi and Sala 2015; Knapp et al. 2017). Altering the
amount of precipitation over many years should provide insight into the
time scales at which water-limited ecosystems respond to chronic
resource alteration. We propose four alternative scenarios for the
effect of precipitation manipulation on the ecosystem functional
response to soil moisture based on the Hierarchical Response Framework
(Fig. 1). We define `ecosystem functional response' as the relationship
between available soil moisture and ANPP. The four scenarios are based
on possible outcomes at the community (e.g., community composition) and
ecosystem (e.g., soil moisture-ANPP regression) levels.

First, altered precipitation might have no effect on either ecosystem
functional response or community composition (Fig. 1, top left). In this
case, changes in ANPP would be well predicted by the current, observed
soil moisture-ANPP relationship. This corresponds to the early phases of
the Hierarchical Response Framework, where ecosystem response follows
the physiological responses of individual species. Second, the ecosystem
functional response might change while community composition remains the
same (Fig. 1, top right). A saturating soil moisture-ANPP response fits
this scenario, where individual species hit physiological thresholds or
are limited by some other resource. Third, the ecosystem functional
response might be constant but community composition changes (Fig. 1,
bottom left). In this case, changes in species' identities and/or
abundances occur in response to altered precipitation levels and species
more suited to the new conditions compensate for reduced function of
initial residents. Fourth, and last, both ecosystem functional response
and community composition could change (Fig. 1, bottom right). New
species, or newly abundant species, with different physiological
responses completely reshape the ecosystem functional response.

All four outcomes are possible in any given ecosystem, but the time
scales at which the different scenarios play out likely differ (Smith,
Knapp, and Collins 2009; Wilcox et al. 2016; Knapp et al. 2017). To
determine these time scales, we need to amass information on how quickly
ecosystem functional responses change in different ecosystems. We also
need to understand whether changes at the ecosystem level are driven by
community level changes or individual level responses.

To that end, here we report the results of a five-year precipitation
manipulation experiment in a sagebrush steppe. We imposed drought and
irrigation treatments (approximately \(\pm50\%\)) and measured ecosystem
(ANPP) and community (species composition) responses. We focus on how
the drought and irrigation treatments affect the relationship between
interannual variation in available soil moisture and interannual
variation in ANPP, and if community dynamics underlie the ecosystem
responses. In particular, we are interested in the consistency of the
soil moisture-ANPP relationship among treatments. Is the relationship
steeper under the drought treatment at low soil moisture? Does the
relationship saturate under the irrigation treatment at high soil
moisture? To answer these questions we fit a generalized linear mixed
effects model to test whether the regressions differed among treatments.
We also analyzed community composition and
\new{the sensitivity of ANPP to drought and irrigation treatments} over
time, allowing us to place our experimental results within the framework
of our scenarios (Fig. 1).

\hypertarget{methods}{%
\section{METHODS}\label{methods}}

\hypertarget{study-area}{%
\subsection{Study Area}\label{study-area}}

We conducted our precipitation manipulation experiment in a sagebrush
steppe community at the USDA-ARS Sheep Experiment Station (USSES) near
Dubois, Idaho (44.2\(^{\circ}\) N, 112.1\(^{\circ}\) W), 1500 m above
sea level. The plant community is dominated by the shrub
\emph{Artemisia tripartita},
\new{the perennial forb \emph{Balsamorhiza sagittata}}, and three
perennial bunchgrasses, \emph{Pseudoroegneria spicata},
\emph{Poa secunda}, and \emph{Hesperostipa comata}
\new{(see Appendix 1 for rank abundance curves)}. During the period of
our experiment (2011 -- 2016), mean annual precipitation was 250 mm
year\(\phantom{}^{-1}\) and mean monthly temperature ranged from
-5.2\(^{\circ}\)C in January to 21.8\(^{\circ}\)C in July. Between 1926
and 1932, range scientists at the USSES established 26 permanent 1
m\(^2\) quadrats to track vegetation change over time. In 2007, we
relocated 14 of the original quadrats in permenanent livestock
exclosures, \new{which we used} as control plots (i.e.~ambient
precipitation) in the experiment described below.
\new{We used the relocated plots as our controls because collecting demographic data is time consuming and it was already being collected in these plots for other studies.}

In spring 2011, we established 16 new 1 m\(^2\) plots located in the
\new{largest exclosure at USSES, which also contained} six of our
control plots. We avoided areas on steep hill slopes, areas with greater
than 20\% cover of bare rock, and areas with greater than 10\% cover of
the shrubs \emph{Purshia tridentata} and/or
\emph{Amelanchier utahensis}. We established the new plots in pairs and
randomly assigned each plot in a pair to receive the ``drought'' or
``irrigation'' treatment.
\new{Thus, our experiment consisted of $n=14$ control plots, $n=8$ irrigation plots, and $n=8$ irrigation plots, for a total of 30 plots.}

\hypertarget{precipitation-experiment}{%
\subsection{Precipitation Experiment}\label{precipitation-experiment}}

Drought and irrigation treatments were designed to decrease and increase
the amount of ambient precipitation by 50\%. To achieve this, we used a
system of rain-out shelters and automatic irrigation (Gherardi and Sala
2013). The rain-out shelters consisted of transparent acrylic shingles
1-1.5 m above the ground that covered an area of \(2.5\times2\) m. The
shingles intercepted approximately 50\% of incoming rainfall, which was
channeled into 75 liter containers. Captured rainfall was then pumped
out of the containers and sprayed on to the adjacent irrigation plot via
two suspended sprinklers. Pumping was triggered by float switches once
water levels reached about 20 liters. We disconnected the irrigation
pumps each October and reconnected them each April. The rain-out
shelters remained in place throughout the year.

We monitored soil moisture in four of the drought-irrigation pairs using
Decagon Devices (Pullman, Washington) 5TM and EC-5 soil moisture
sensors. We installed four sensors around the edges of each 1x1 m census
plot, two at 5 cm soil depth and two at 25 cm soil depth. We also
installed four sensors in areas nearby the four selected plot pairs to
measure ambient soil moisture at the same depths. Soil moisture
measurements were automatically logged every four hours. We coupled this
temporally intensive soil moisture sampling with spatially extensive
readings \new{taken with a handheld EC-5 sensor} at six points within
all 16 plots and associated ambient measurement areas. These snapshot
data were collected on 06-06-2012, 04-29-2015, 05-07-2015, 06-09-2015,
and 05-10-2016\footnote{Dates formatted as: mm-dd-yyyy.}.

Analyzing the response to experimental treatments was complicated by the
fact that we did not directly monitor soil moisture in each plot on each
day of the experiment. Only a subset of plots were equipped with soil
moisture sensors, and within those plots, one or more of the sensors
frequently failed to collect data. Therefore, we used a statistical
model to estimate average daily soil moisture values for the ambient,
drought, and irrigation treatments during the course of the experiment.

\new{We first averaged the observed soil moisture readings for each day ($d$) and plot ($i$), $x_{i,d}$.
Experimental plots were located in pairs, with each group ($g$) containing a drought and irrigated plot. 
In addition, a nearby area outside the drought or irrigated plots was monitored for local ambient soil moisture conditions.
Within each group of plots we standardized the irrigation and drought effects on soil moisture relative to the ambient soil moisture conditions. 
Specifically, we subtracted the ambient daily soil moisture from the soil moisture measured within the drought and irrigation plots within each group and then divided by the standard deviation of daily soil moisture values measured in the ambient conditions $\left( \Delta x_{g,d,\text{trt.}} = (x_{g,d,\text{trt.}} - x_{g,d,\text{ambient}})/\text{s.d.}(x_{g,d,\text{ambient}}) \right)$ where $\Delta x$ is the standardized treatment effect and $x_{g,d,\text{trt.}}$ is the raw soil moisture measure for plot group $g$ on day $d$ and treatment $trt$.
These transformations ensured that the treatment effects in each plot were appropriately scaled by the local ambient conditions within each plot group.}

\new{We then modeled these daily deviations ($\Delta x_{g,d,\text{trt.}}$)}
from ambient conditions using a linear mixed effects model with
independent variables for treatment (irrigation or drought), season
(winter, spring, summer, fall), rainfall, and all two-way interactions.
Using the local daily weather station data, we recorded rainy days as
any day with measureable precipitation or the day after such a day and
with average temperatures above 3\(^{\circ}\)C (to exclude days with
snowfall). We fit the model using the \texttt{lme4::lmer()} function
(Bates et al. 2015) in R (R Core Team 2016), with random effects for
plot group and date. We weighted observations by the number of unique
sensors or spot measurements that were taken in each plot on that day.
\new{We then used the model to predict the deviations from ambient conditions produced in the treated plots on each day of the experiment.  
We added these predicted deviations to the average daily ambient soil moisture to generate predictions for daily soil moisture in all of the treated plots: $\bar{x}_{d, \text{trt.}} = \Delta \bar{x}_{d,\text{trt.}} + \bar{x}_{d,\text{ambient}}$, where $\bar{x}_{d,\text{trt.}}$ is the average predicted daily soil moisture in the treated plots and $\bar{x}_{d,\text{ambient}}$ is the daily ambient soil moisture averaged across all control plots.}
We could only predict soil moisture in the treated plots on days for
which we took at least one ambient soil moisture measurement. This
procedure allowed us to predict daily soil moisture conditions for all
plots, even on those days when some of our direct treatment measurements
were missing due to malfunction of the sensors or data loggers. See
Kleinhesselink (2017) for more details on our approach to estimating
soil moisture.

\new{Following the above procedure, we still lacked soil moisture data for March 2012 (observations did not start unitl April) and for a string of days in 2013 during which the soil moisture probes failed to take readings.
To fill in these gaps, we used a version of the SOILWAT soil moisture model}
(Sala, Lauenroth, and Parton 1992)
\new{that has been specifically designed for semiarid shrublands and grasslands}
(Bradford, Schlaepfer, and Lauenroth 2014).
\new{The model was parameterized using generic sagebrush steppe parameters and local soil texture, soil bulk density, and precipitation data}
(Kleinhesselink 2017).
\new{We used the locally parameterized SOILWAT model to generate daily soil moisture predictions for the duration of the experiment, but only used SOILWAT predictions where there were gaps in our data (Fig. 2B and Appendix 2).
SOILWAT predicted daily soil moisture under ambient conditions similar to our control plots.
We applied the same statistical model and procedure described above to estimate soil moisture in drought and irrigation plots based on control plot conditions.}

\hypertarget{data-collection}{%
\subsection{Data Collection}\label{data-collection}}

We estimated aboveground net primary productivity (ANPP) using a
radiometer to relate ground reflectance to plant biomass (Byrne et al.
2011). We recorded ground reflectance at four wavelengths, two
associated with red reflectance (626 nm and 652 nm) and two associated
with near-infrared reflectance (875 nm and 859 nm). At each plot in each
year, we took four readings of ground reflectances at the above
wavelengths. We also took readings in 12 (2015), 15 (2012, 2013, 2014),
or 16 (2016) calibration plots adjacent to the experimental site, in
which we harvested all aboveground biomass produced in the current year
(we excluded litter and standing dead material), dried it to a constant
weight at 60\(^{\circ}\)C, and weighed it to estimate ANPP. We made
radiometer measurements and harvested at peak green biomass each year,
typically in late June.

For each plot and year, we averaged the four readings for each
wavelength and then calculated a greenness index based on the same bands
used to calculate NDVI using the MODIS (Moderate Resolution Imaging
Spectroradiometer) and AVHRR (Advanced Very High Resolution Radiometer)
bands for NDVI. We regressed the greenness index against the dry biomass
weight from the calibration plots to convert the greenness index to
ANPP. We fit regressions to a MODIS-based index and an AVHRR-based index
for each year and retained the regression with the better fit based on
\(R^2\) values. We then predicted ANPP using the best regression
equation for each year (Appendix 3).
\new{Our results do not change when we analyze the soil moisture-NDVI relationship instead of the soil moisture-ANPP relationship (Appendix 5).}

Species composition data came from two sources: yearly census maps for
each plot made using a pantograph (Hill 1920) and yearly counts of
annual species in each plot. From these sources, we determined the
density of all annuals and perennials forbs, the basal cover of
perennial grasses, and the canopy cover of shrubs. We made a large
plot-treatment-year by species matrix, where columns were filled with
either cover or density, depending on the measurement made for the
particular species. We standardized the values in each column so we
could directly compare species quantified with different metrics
(density, basal cover, and canopy cover). This puts all abundance values
on the same scale, meaning that common and rare species are weighted
equally. Assuming that rare species will respond to treatments more than
common ones, our approach is biased towards detecting compositional
changes.
\new{We limited our analysis of community data to observations from the permanent exclosure containing our drought and irrigation treatments (see \textbf{Community composition over time}).
This exclosure included six of the control plots and all of the treatment plots, for a total of 22 plots.
The other eight control plots are in other pastures. 
Including these plots would add spatial variation in composition, complicating our goal of describing temporal trends in composition.}

\hypertarget{data-analysis}{%
\subsection{Data Analysis}\label{data-analysis}}

\hypertarget{ecosystem-functional-response}{%
\subsubsection{Ecosystem functional
response}\label{ecosystem-functional-response}}

Our main goal was to test whether the relationship between ANPP and soil
moisture differed among the drought, control, and irrigation treatments.
Based on our own observations and previous work at our study site
(Blaisdell 1958; Dalgleish et al. 2011; Adler, Dalgleish, and Ellner
2012), we chose to use cumulative volumetric water content from March
through June as our metric of soil moisture (hereafter referred to as
`VWC'). We fit a generalized linear mixed effects regression model with
log(ANPP) as the response variable and VWC and treatment as fixed
effects. Plot and year of treatment were included as random effects to
account for non-independence of observations, as described below. We
log-transformed ANPP to reduce heteroscedasticity. Both log(ANPP) and
VWC were standardized to have mean 0 and unit variance before fitting
the model {[}i.e., \((x_i - \bar{x})/\sigma_x\){]}.

Our model is defined as follows:

\vspace{-2em}

\begin{align}
\mu_{i} &= \boldsymbol{\beta}\textbf{x}_i + \boldsymbol{\gamma}_{j(i)}\textbf{z}_i + \eta_t, \\
\textbf{y} &\sim \text{Normal} \left(\boldsymbol{\mu}, \sigma^2 \right),
\end{align}

\noindent{}where \(\mu_{i}\) is the deterministic prediction from the
regression model for observation \(i\), which is associated with plot
\(j\) and treatment year \(t\). \(\boldsymbol{\beta}\) is the vector of
coefficients for the fixed effects in the design matrix \(\textbf{X}\).
Each row of the design matrix represents a single observation
(\(\textbf{x}_i\)) and is a vector with the following elements: 1 for
the intercept, a binary 0 or 1 if the treatment is ``drought'', a binary
0 or 1 if the treatment is ``irrigation'', the scaled value of VWC,
binary ``drought'' value times VWC, and binary ``irrigation'' value
times VWC. Thus, our model treats ``control'' observations as the main
treatment and then estimates intercept and slope offsets for the
``drought'' and ``irrigation'' treatments. We use our model to test two
statistical hypotheses based on the questions in our Introduction:

\begin{description}
\item [H1.] The coefficient for drought$\times$VWC is positive and different from zero.
\item [H2.] The coefficient for irrigation$\times$VWC is negative and different from zero.
\end{description}

\noindent{}These hypotheses are based on evidence that
precipitation-ANPP relationships often saturate with increasing
precipitation (Hsu, Powell, and Adler 2012; Laureano A Gherardi and Sala
2015).

We include two random effects to account for the fact that observations
within plots and years are not independent. Specifically, we include
plot-specific offsets (\(\boldsymbol{\gamma}\)) for the intercept and
slope terms and year-specific intercept offsets (\(\eta_t\)). The
covariate vector \(\textbf{z}_i\) for each observation \(i\) has two
elements: a 1 for the intercept and the scaled value of VWC for that
plot and year. The plot-specific coefficients are modeled
hierarchically, where plot level coefficients are drawn from a
multivariate normal distribution with mean 0 and a variance-covariance
structure that allows the intercept and slope terms to be correlated:

\vspace{-1em}

\begin{align}
\boldsymbol{\gamma}_{j(i)} &\sim \text{MVN} \left( 0, \Sigma  \right),
\end{align}

\noindent{}where \(\Sigma\) is the variance-covariance matrix and
\(j(i)\) reads as ``plot \(j\) associated with observation \(i\)''. The
random year effects (\(\boldsymbol{\eta}\)) are drawn from a normal
prior with mean 0 and standard deviation \(\sigma_{\text{year}}\), which
was drawn from a half-Cauchy distribution.
\new{We used a vague prior distribution for each $\beta$: $\boldsymbol{\beta} \sim \text{Normal}(0,5)$.}
A full description of our model and associated R (R Core Team 2016) code
is in Appendix 4.

We fit the model using a Bayesian approach and obtained posterior
estimates of all unknowns via the No-U-Turn Hamiltonian Monte Carlo
sampler in Stan (Stan Development Team 2016b). We used the R package
`rstan' (Stan Development Team 2016a) to link R (R Core Team 2016) to
Stan. We obtained samples from the posterior distribution for all model
parameters from four parallel MCMC chains run for 10,000 iterations,
saving every \(10^{\text{th}}\) sample. Trace plots of all parameters
were visually inspected to ensure well-mixed chains and convergence. We
also made sure all scale reduction factors (\(\hat{R}\) values) were
less than 1.1 (Gelman and Hill 2009).

\hypertarget{sensitivity-of-anpp-over-time}{%
\subsubsection{Sensitivity of ANPP over
time}\label{sensitivity-of-anpp-over-time}}

\new{Our data do not allow us to directly test whether the ecosystem functional response to precipitation may have changed over time since the start of the experiment in each of the treatments. 
This is because we lack sufficient within-year and within-treatment variation of soil moisture (i.e., within a year and treatment each plot shares the same value of VWC). 
However, we did use a separate analysis to test whether the the sensitivity of ANPP to treatment changed over time.}

\new{We define `sensitivity' as $\frac{ \text{ANPP}_{\text{control}} - \text{ANPP}_{\text{treatment}} } { \text{VWC}_{\text{ambient}} - \text{VWC}_{\text{treatment}} }$, following}
Wilcox et al. (2017).
\new{This metric of sensitivity scales the difference between ANPP in the treated and control plots by the change in soil moisture in the treated plots. 
Because our treated plots are not directly paired with individual control plots, we compare ANPP in each treated plot to the average ANPP of the control plots in each year.
After calculating sensitivity for each drought and irrigation plot in each year, we regressed sensitivity against year of treatment using the \texttt{lm()} function in R}
(R Core Team 2016).
\new{This analysis also allows us to link particularly sensitive treatment-years to changes in community composition.}

\hypertarget{community-composition-over-time}{%
\subsubsection{Community composition over
time}\label{community-composition-over-time}}

We used nonmetric multidimensional scaling (NMDS) based on Bray-Curtis
distances to identify temporal changes in community composition among
treatments. We first calculated Bray-Curtis distances among all plots
for each year of the experiment and then extracted those distances for
use in the NMDS. Some values of standardized species' abundances were
negative, which is not allowed for calculating Bray-Curtis distances. We
simply added `2' to each abundance value to ensure all values were
greater than zero. We plotted the first two axes of NMDS scores to see
if community composition overlapped, or not, among treatments in each
year. We used the \texttt{vegan::metaMDS()} function (Oksanen 2016) to
calculate Bray-Curtis distances and then to run the NMDS analysis. We
used the \texttt{vegan::adonis()} function (Oksanen 2016) to perform
permutational multivariate analysis of variance to test whether
treatment plots formed distinct groupings. To test whether treatment
plots were equally dispersed, or not, we used the
\texttt{vegan::betadisper()} function (Oksanen 2016).

\new{We conducted the above analysis for all species, and then conducted a separate analysis for annual species only (Appendix 1). 
Annual species have shorter life spans, so conducting a separate analysis allowed us to test whether we might find stronger evidence for community responses to altered precipitation when we focus on short-lived species. 
Given the dominance of perennial species in our system (Appendix 1), shifts in the annual plant community could be difficult to detect in the analysis of the full community.}

\hypertarget{reproducibility}{%
\subsubsection{Reproducibility}\label{reproducibility}}

All R code and data necessary to reproduce our analysis has been
archived on Figshare (\emph{link here after acceptance}) and released on
GitHub (\url{https://github.com/atredennick/usses_water/releases/v0.1}).
We also include annotated Stan code in our model description in Appendix
4.

\hypertarget{results}{%
\section{RESULTS}\label{results}}

Ambient precipitation and soil moisture were variable over the five
years of the study (Fig. 2A-B). ANPP varied from a minimum of 77.4 g
m\(^{-2}\) in 2014 to a maximum of 239.3 g m\(^{-2}\) in 2016 when
averaged across treatments (Fig. 2C). ANPP was slightly higher in
irrigation plots \new{(on average 15\% higher)} and slightly lower in
drought plots \new{(on average 25\% lower) relative to control plots}
(Fig. 2C), corresponding to observed and estimated soil volumetric water
content (VWC) differences among treatments (Fig. 2B).
\new{March-June VWC in drought plots was 12\% less than in control plots on average, and March-June VWC in irrigation plots was 19\% higher than in control plots on average across the years of the experiment.}
The differences in soil VWC indicate our treatment infrastructure was
successful. ANPP was highly variable across plots within years,
\new{as indicated by the large and overlapping standard deviations}
(Fig. 2C).

Cumulative March-June soil moisture had a weak positive effect on ANPP
(Table 1; Fig. 3B). The effect of soil moisture for each treatment is
associated with high uncertainty, however, with 95\% Bayesian credible
intervals that \new{broadly} overlap zero (Table 1). Although the
parameter estimates for the effect of soil moisture overlap zero, the
posterior distributions of the slopes all shrank and shifted to more
positive values relative to the prior distributions (Fig. A3-2), which
indicates the data did influence parameter estimates. Ecosystem
functional response was similar among treatments (Table 1; Fig. 3B), but
there is evidence that the
\new{slope for the drought treatment is greater than the slope for the control treatment.}
This evidence comes from interpreting the posterior distributions of the
slope offsets for the treatments. \new{From these distributions,} we
calculate a \new{42}\% one-tailed probability that the estimate is less
than zero for the irrigation treatment
\new{and a 100\% one-tailed probability that the estimate is greater than zero for the drought treatment}
(Fig. 3A, right panel).

\new{Sensitivity of ANPP to irrigation was constant over the course of our experiment (Fig. 3C). 
Sensivitity of ANPP to drought, however, grew over time (\emph{P} = 0.0007; Fig. 3C)}.

Community composition was similar among treatments. The multidimensional
space of community composition overlapped among treatments in all years
and was equally dispersed in all years (Table 2; Fig. 4). Community
composition was also remarkably stable over time, with no evidence of
divergence among treatments (Table 2; Fig. 4).
\new{There were also no changes in the dominant species over time in any treatment (Appendix 1).
Analyzing annual species on their own produced similar results.
There is some evidence that the annual community changed in response to treatment in two years (Table A1-1; Fig. A1-5), but these responses appear to come from very rare species since our analysis weights common and rare species equally (Fig. A1-4).
By definition, these rare species contribute little to ANPP and thus any compositional changes have little influence on ecosystem functional response.}

\hypertarget{discussion}{%
\section{DISCUSSION}\label{discussion}}

Ecosystem response to a new precipitation regime depends on the
physiological responses of constituent species and the rate at which
community composition shifts to favor species better able to take
advantage of, or cope with, new resource levels (Smith, Knapp, and
Collins 2009). Previous work has shown that community compositional
shifts can be either rapid, on the order of years (Hoover, Knapp, and
Smith 2014), or slow, on the order of decades (Knapp, Briggs, and Smith
2012; Wilcox et al. 2016). A lingering question is how the time scales
of ecosystem response and community change vary among ecosystems.
Precipitation manipulation experiments can help answer this question,
especially if they push water availability outside the historical range
of variability for long periods.

We found that ecosystem functional response under chronic \new{drought}
was different from the control treatment, but community composition
remained unchanged (Fig. 3A, Fig. 4), representing the top right
scenario in Fig. 1.
\new{The increase in the slope of the VWC $\times$ productivity relationship in the drought plots indicates increased sensitivity to water availability under chronic drought.
A strict interpretation of this result implies that if average soil moisture were pushed consistently lower than currently observed ambient conditions, there would be a stronger relationship between precipitation and productivity in this system.}

\new{However, we do not want to over-interpret the significance of the slope offsets given that the overall slopes}
of the VWC \(\times\) productivity relationship,
\new{not just the offsets}, were similar among treatments (Table 1). We
therefore conclude that ecosystem functional response is consistent
(\new{similar values}) and weak (\new{all broadly overlapping zero})
across all precipitation treatments.

The similarity of ecosystem functional response (Fig. 3) and community
composition (Fig. 4) among treatments is surprising because grasslands
generally, and sagebrush steppe specifically, are considered
water-limited systems. \new{For example,} Huxman et al. (2004) and Knapp
et al. (2015)
\new{showed that semi-arid sites are more sensitive to drought than mesic sites, and}
Wilcox et al. (2017)
\new{found that semi-arid sites are particularly sensitive to irrigation treatments.}
\new{Based on these findings}, we expected ecosystem functional
response, community composition, or both to change under our treatments.
Why did our treatments fail to induce ecosystem or community responses?
We can think of three reasons. Two are limitations of our study, and one
invokes the life history traits of the species at our study site.

First, our precipitation manipulations may not have been large enough to
induce a response. A 50\% decrease in any given year may not be abnormal
given our site's historical range of variability (Knapp et al. 2017). We
cannot definitively rule out this possibility, but we have reason to
believe our treatments \emph{should} have been large enough. Using the
methods described by Lemoine et al. (2016), we calculated the percent
reduction and increase of \new{water-year} precipitation necessary to
reach the 1\% and 99\% extremes of the historical precipitation regime
at our site (Fig. A5-1).
\new{The 1\% quantile of water-year precipitation at our site is 78 mm, a 26\% reduction from the mean, and the 99\% quantile is 545 mm, a 84\% increase from mean growing season precipitation (Appendix 6).
Thus, our drought treatment represented extreme precipitation amounts, and that is the treatment where we observed a small effect on the slope between soil moisture and ANPP (Fig. 3A).
The irrigation treatment may not have been extreme enough to induce a response.}

Second, ANPP at our site is likely influenced by additional factors, not
only the cumulative March-June soil moisture covariate we included in
our statistical model. \new{For example,} La Pierre et al. (2016)
\new{found that site-scale ANPP is better predicted by nutrient availability than precipitation.}
Moreover, temperature can impact ANPP directly (Epstein, Lauenroth, and
Burke 1997) and by exacerbating the effects of soil moisture (De Boeck
et al. 2011). Measurements of soil moisture likely contain a signal of
temperature, through its impact on evaporation and infiltration, but the
measurements will not capture the direct effect of temperature on
metabolic and physiological processes. We also did not redistribute snow
across our treatments in the winter, and snow melt may spur early spring
growth. Failure to account for potentially important covariates could
explain the within-year spread of ANPP (Fig 2C, Fig. 3B) and the
resulting uncertain relationship we observed between soil moisture and
ANPP across all treatments (Table 1, Fig. A3-2).

Third, the life history traits of the dominant species in our study
ecosystem may explain the consistently positive, but weak and uncertain,
effect of soil moisture on ANPP (Table 1, Fig. 3). Species that live in
variable environments, such as cold deserts, must have strategies to
ensure long-term success as conditions vary. One strategy is bet
hedging, where species forego short-term gains to reduce the variance of
long-term success (Seger 1987). In other words, species follow the same
conservative strategy every year, designed to minimize losses during
unfavorable periods. The dry and variable environment of the sagebrush
steppe has likely selected for bet hedging species that can maintain
function at low water availability and have weak responses to high water
availability. In so doing, the dominant species in our ecosystem avoid
``boom and bust'' cycles, which may explain the weak effect of soil
moisture on ANPP (i.e., the Bayesian credible intervals for the slopes
overlapping zero; Table 1).

Another strategy to deal with variable environmental conditions is
avoidance, which would also result in a consistent ecosystem functional
response between drought and control treatments. For example, many of
the perennial grasses in our focal ecosystem avoid drought stress by
growing early in the growing season (Blaisdell 1958, A.R.
Kleinhesselink, personal observation). Furthermore, the dominant shrub
in our focal ecosystem, \emph{Artemisia tripartita}, has access to water
deep in the soil profile thanks to a deep root system (Kulmatiski et al.
2017).
\new{The dominance of our site by the shrub \emph{A. tripartita} (Appendix 1) may explain why our results do not conform to broader patterns of grassland sensitivity to precipitation manipulations}
(e.g., Huxman et al. 2004; Knapp et al. 2015; Wilcox et al. 2017).

\new{We found that ANPP became more sensitive to the drought treatment over time (Fig. 3C).
We interepret this increase in sensitivity as the effect of cumulative impacts of the drought on dominant species.
Plants may not have shown a large response in terms of ANPP in the first years of the experiment if they could grow from stored carbohydrate reserves or if they abstained from flowering and reproduction.
As the drought progressed, these same plants may have have started to shrink or die. 
Given the long-lived perennial species at this site, this increase in sensitivity may indicate a larger future change in community composition and ecosystem functional response. }

\hypertarget{conclusions}{%
\section{CONCLUSIONS}\label{conclusions}}

Our results suggest that the species in our focal plant community are
insensitive to to changes in precipitation regime,
\new{at least over the five years of our experiment}. Such insensitivity
can buffer species against precipitation variability in this semi-arid
ecosystem, making them successful in the long run. Longer, chronic
precipitation alteration might reveal plant community shifts that we did
not observe (e.g., Wilcox et al. 2016),
\new{and the increased sensitivity of ANPP to drought in the final years of our experiment may portend such a shift.}
Likewise, a long-term increase in water availability could allow species
that do not bet hedge to gain prominence and dominate the ecosystem
functional response.
\new{The length of the perturbation may be especially relevant in our focal ecosystem where the perennial species are long-lived, meaning compositional turnover may take many more years than we report on here.}

\hypertarget{acknowledgements}{%
\section{ACKNOWLEDGEMENTS}\label{acknowledgements}}

We thank the many summer research technicians who collected the data
reported in this paper and the USDA-ARS Sheep Experiment Station
(Dubois, ID) for facilitating work on their property. We also thank
Susan Durham for clarifying our thinking on the statistical analyses and
Kevin Wilcox for helpful discussions on analyzing community composition
data.
\new{John Bradford and Caitlin Andrews generously provided SOILWAT model output.}
\new{Two anonymous reviewers and Elsie Denton provided thoughtful reviews that improved our paper.}

\hypertarget{funding}{%
\section{FUNDING}\label{funding}}

This research was supported by the Utah Agricultural Experiment Station,
Utah State University, and approved as journal paper number 9035. The
research was also supported by the National Science Foundation, through
a Postdoctoral Research Fellowship in Biology and Mathematics to ATT
(DBI-1400370), a Graduate Research Fellowship to ARK, and grants
DEB-1353078 and DEB-1054040 to PBA.

\hypertarget{author-contributions}{%
\section{AUTHOR CONTRIBUTIONS}\label{author-contributions}}

\begin{itemize}
  \item Andrew T. Tredennick collected data, analyzed the data, wrote the paper, prepared figures and/or tables, reviewed drafts of the paper.
  \item Andrew R. Kleinhesselink conceived and designed the experiments, performed the experiments, collected data, analyzed the data, reviewed drafts of the paper.
  \item J. Bret Taylor contributed reagents/materials/analysis tools, reviewed drafts of the paper.
  \item Peter B. Adler conceived and designed the experiments, performed the experiments, collected data, analyzed the data, reviewed drafts of the paper.
\end{itemize}

\hypertarget{supplemental-information}{%
\section{SUPPLEMENTAL INFORMATION}\label{supplemental-information}}

\begin{description}
\item [Appendix 1.] \new{Additional information on plant community composition and dynamics, Table A1-1, Fig. A1-1, Fig. A1-2, Fig. A1-3, Fig. A1-4, and Fig. A1-5.}
\item [Appendix 2.] \new{Additional details on soil moisture modeling with SOILWAT and Fig. A2-1.}
\item [Appendix 3.] Additional methods and information on estimating aboveground net primary productivity, Table A3-1, and Table A3-2.
\item [Appendix 4.] Details of the hierarchical Bayesian model, Fig. A4-1, Fig. A4-2, and Fig. A4-3.
\item [Appendix 5.] \new{Results from analysis of NDVI without conversion to ANPP, Fig. A5-1.}
\item [Appendix 6.] Details on analysis of precipitation historical range of variability and Fig. A6-1.
\end{description}

\newpage{}

\hypertarget{tables}{%
\section{TABLES}\label{tables}}

\begin{table}[ht]
\centering
\caption{Summary statistics from the posterior distributions of coefficients for each treatment ($\beta$ coefficients in equation 1). The `Intercept' and `Slope' summaries reported here for drought and irrigation are from the posterior distributions of the intercept and slope for the control treatment plus the offsets for each treatment. Posterior distributions of the offsets are in Figure 3A.} 
\begingroup\normalsize
\begin{tabular}{llrrrr}
  \hline
Coefficient & Treatment & Posterior Mean & Posterior Median & Lower 95\% BCI & Upper 95\% BCI \\ 
  \hline
Intercept & Control & 0.14 & 0.14 & -1.24 & 1.50 \\ 
  Intercept & Drought & 0.67 & 0.65 & -2.01 & 3.36 \\ 
  Intercept & Irrigation & -0.39 & -0.34 & -3.51 & 2.48 \\ 
  Slope & Control & 0.53 & 0.52 & -1.44 & 2.66 \\ 
  Slope & Drought & 1.02 & 1.00 & -1.04 & 3.25 \\ 
  Slope & Irrigation & 0.56 & 0.55 & -1.43 & 2.71 \\ 
   \hline
\end{tabular}
\endgroup
\end{table}\begin{table}[ht]
\centering
\caption{Results from statistical tests for clustering and dispersion of community composition among precipitation treatments. `adonis' tests whether treatments form unique clusters in multidimensional space; `betadisper' tests whether treatments have similar dispersion. For both tests, \emph{P} values greater than 0.05 indicate there is no support for the hypothesis that the treatments differ.} 
\begingroup\normalsize
\begin{tabular}{rlrrrr}
  \hline
Year & Test & n & d.f. & \emph{F} & \emph{P} \\ 
  \hline
2011 & adonis &  22 &   2 & 1.04 & 0.43 \\ 
  2011 & betadisper &  22 &   2 & 2.22 & 0.14 \\ 
  2012 & adonis &  22 &   2 & 1.09 & 0.32 \\ 
  2012 & betadisper &  22 &   2 & 0.21 & 0.81 \\ 
  2013 & adonis &  22 &   2 & 1.26 & 0.12 \\ 
  2013 & betadisper &  22 &   2 & 0.43 & 0.66 \\ 
  2014 & adonis &  22 &   2 & 0.96 & 0.58 \\ 
  2014 & betadisper &  22 &   2 & 0.36 & 0.70 \\ 
  2015 & adonis &  22 &   2 & 1.08 & 0.33 \\ 
  2015 & betadisper &  22 &   2 & 1.96 & 0.17 \\ 
  2016 & adonis &  22 &   2 & 1.12 & 0.26 \\ 
  2016 & betadisper &  22 &   2 & 0.27 & 0.76 \\ 
   \hline
\end{tabular}
\endgroup
\end{table}

\newpage{}

\hypertarget{figures}{%
\section{FIGURES}\label{figures}}

\begin{figure}[!ht]
  \centering
      \includegraphics[width=4in]{../figures/hypothesis_figtable.png}
  \caption{Possible outcomes of chronic resource alteration based on the 'Hierarchical Response Framework' (Smith et al. 2009).}
\end{figure}

\newpage{}

\begin{figure}[!ht]
  \centering
      \includegraphics[height=7in]{../figures/data_panels.png}
  \caption{(A) Probability density of historical water-year (Oct.--Sept.) precipitation from 1927-2016, with the years of the experiment shown by arrows on the \emph{x}-axis. (B) \new{Statistically estimated (solid lines) and SOILWAT-generated (dashed lines) daily} soil volumetric water content (VWC) in each of the three treatments during the course of the experiment. \new{We used the estimates from our statistical model, with gaps filled in by SOILWAT predictions, to calculate cumulative March--June VWC as used in our analysis of treatment effects on ecosystem functional response.} (C) Mean (points) ANPP and its standard deviation (error bars) for each year of the experiment. Colors in panels B and C identify the treatment, as specified in the legend of panel C.}
\end{figure}

\newpage{}

\begin{figure}[!ht]
  \centering
      \includegraphics[width=5in]{../figures/glmm_main_results.png}
  \caption{Results from the generalized linear mixed effects model \new{(A-B) and the sensitivity analysis (C)}. (A) Posterior distributions for the effects of drought and irrigation on the intercept and slope of the productivity-soil moisture relationship. Treatment effects show the difference between the coefficients estimated in the treated plots and the control plots. Probabilities (``Pr $</>$ 0 ='') for each coefficient indicate the one-tailed probability that the coefficient is less than or greater than zero, depending on the median of the intercept offset distributions or our specific hypothesis for the slope offsets. The posterior densities were smoothed for visual clarity by increasing kernel bandwidth by a factor of five. (B) Scatterplot of the data and model estimates shown as solid lines. Model estimates come from treatment level coefficients (colored lines). Note that we show log(ANPP) on the \emph{y}-axis of panel B; this same plot can be seen on the arithmetic scale in supporting material Fig. A2-1. \new{(C) Regression of sensitivity against time for each treatment. Each point represents the sensitivity of ANPP in a plot relative to the mean of the control plots in that year. Only the significant regression for the drought treatment is plotted (\emph{P} = 0.0007).}}
\end{figure}

\newpage{}

\begin{figure}[!ht]
  \centering
      \includegraphics[width=5in]{../figures/sppcomp_bray_all.png}
  \caption{Nonmetric multidimensional scaling scores representing plant communities in each plot, colored by treatment. \new{Each point represents a plot and the axes represent species composition.}}
\end{figure}

\newpage{}

\hypertarget{references}{%
\section*{REFERENCES}\label{references}}
\addcontentsline{toc}{section}{REFERENCES}

\hypertarget{refs}{}
\leavevmode\hypertarget{ref-Adler2012}{}%
Adler, P B, H J Dalgleish, and S P Ellner. 2012. ``Forecasting plant
community impacts of climate variability and change: when do competitive
interactions matter?'' \emph{Journal of Ecology} 100:478--87.
\url{https://doi.org/10.1111/j.1365-2745.2011.01930.x}.

\leavevmode\hypertarget{ref-Avolio2015}{}%
Avolio, Meghan L, Kimberly J La Pierre, Gregory R Houseman, Sally E
Koerner, Emily Grman, Forest Isbell, David Samuel Johnson, and Kevin R
Wilcox. 2015. ``A framework for quantifying the magnitude and
variability of community responses to global change drivers.''
\emph{Ecosphere} 6 (12):1--14.
\url{https://doi.org/10.1890/ES15-00317.1}.

\leavevmode\hypertarget{ref-Bates2015}{}%
Bates, D., M. Maechler, B. Bolker, and S. Walker. 2015. ``Fitting linear
mixed-effects models using lme4.''
\url{https://doi.org/10.18637/jss.v067.i01}.

\leavevmode\hypertarget{ref-Blaisdell1958}{}%
Blaisdell, James P. 1958. ``Seasonal development and yield of native
plants on the upper snake river plains and their relation to certain
climate factors.'' \emph{United States Department of Agriculture
Technical Bulletin No. 1190}.

\leavevmode\hypertarget{ref-Bradford2014}{}%
Bradford, John B., Daniel R. Schlaepfer, and William K. Lauenroth. 2014.
``Ecohydrology of Adjacent Sagebrush and Lodgepole Pine Ecosystems: The
Consequences of Climate Change and Disturbance.'' \emph{Ecosystems} 17
(4):590--605. \url{https://doi.org/10.1007/s10021-013-9745-1}.

\leavevmode\hypertarget{ref-Byrne2011}{}%
Byrne, Kerry M., William K. Lauenroth, Peter B. Adler, and Christine M.
Byrne. 2011. ``Estimating Aboveground Net Primary Production in
Grasslands: A Comparison of Nondestructive Methods.'' \emph{Rangeland
Ecology and Management} 64 (5):498--505.
\url{https://doi.org/10.2111/REM-D-10-00145.1}.

\leavevmode\hypertarget{ref-Dalgleish2011}{}%
Dalgleish, Harmony J., David N. Koons, Mevin B. Hooten, Corey A. Moffet,
and Peter B. Adler. 2011. ``Climate influences the demography of three
dominant sagebrush steppe plants.'' \emph{Ecology} 92 (1):75--85.
\url{https://doi.org/10.1890/10-0780.1}.

\leavevmode\hypertarget{ref-DeBoeck2011}{}%
De Boeck, Hans J., Freja E. Dreesen, Ivan A. Janssens, and Ivan Nijs.
2011. ``Whole-system responses of experimental plant communities to
climate extremes imposed in different seasons.'' \emph{New Phytologist}
189 (3):806--17. \url{https://doi.org/10.1111/j.1469-8137.2010.03515.x}.

\leavevmode\hypertarget{ref-Epstein1997}{}%
Epstein, H. E., W. K. Lauenroth, and I. C. Burke. 1997. ``Effects of
temperature and soil texture on ANPP in the U.S. Great plains.''
\emph{Ecology} 78 (8):2628--31.
\href{https://doi.org/10.1890/0012-9658(1997)078\%5B2628:EOTAST\%5D2.0.CO;2}{https://doi.org/10.1890/0012-9658(1997)078{[}2628:EOTAST{]}2.0.CO;2}.

\leavevmode\hypertarget{ref-Gelman2009}{}%
Gelman, Andrew, and Jennifer Hill. 2009. \emph{Data analysis using
regression and multilevel/hierarchical models}. Cambridge: Cambridge
University Press.

\leavevmode\hypertarget{ref-Gherardi2015a}{}%
Gherardi, Laureano A, and Osvaldo E Sala. 2015. ``Enhanced precipitation
variability decreases grass- and increases shrub-productivity.''
\emph{Proceedings of the National Academy of Sciences} 112
(41):12735--40. \url{https://doi.org/10.1073/pnas.1506433112}.

\leavevmode\hypertarget{ref-Gherardi2013}{}%
Gherardi, Laureano A., and Osvaldo E Sala. 2013. ``Automated rainfall
manipulation system: a reliable and inexpensive tool for ecologists.''
\emph{Ecosphere} 4 (2):1--10.
\url{https://doi.org/10.1890/ES12-00371.1}.

\leavevmode\hypertarget{ref-Gherardi2015}{}%
Gherardi, Laureano A., and Osvaldo E. Sala. 2015. ``Enhanced interannual
precipitation variability increases plant functional diversity that in
turn ameliorates negative impact on productivity.'' \emph{Ecology
Letters} 18 (12):1293--1300. \url{https://doi.org/10.1111/ele.12523}.

\leavevmode\hypertarget{ref-Hill1920}{}%
Hill, Robert R. 1920. ``Charting Quadrats with a Pantograph.''
\emph{Ecology} 1 (4). Ecological Society of America:270--73.
\url{https://doi.org/10.2307/1929561}.

\leavevmode\hypertarget{ref-Hoover2014}{}%
Hoover, David L., Alan K. Knapp, and Melinda D. Smith. 2014.
``Resistance and resilience of a grassland ecosystem to climate
extremes.'' \emph{Ecology} 95 (9):2646--56.
\url{https://doi.org/10.1890/13-2186.1}.

\leavevmode\hypertarget{ref-Hsu2012}{}%
Hsu, Joanna S., James Powell, and Peter B. Adler. 2012. ``Sensitivity of
mean annual primary production to precipitation.'' \emph{Global Change
Biology} 18 (7):2246--55.
\url{https://doi.org/10.1111/j.1365-2486.2012.02687.x}.

\leavevmode\hypertarget{ref-Huxman2004}{}%
Huxman, Travis E., Melinda D. Smith, Philip A. Fay, Alan K. Knapp, M.
Rebecca Shaw, Michael E. Loik, Stanley D. Smith, et al. 2004.
``Convergence across biomes to a common rain-use efficiency.''
\emph{Nature} 429 (6992):651--54.
\url{https://doi.org/10.1038/nature02561}.

\leavevmode\hypertarget{ref-Kleinhesselink2017b}{}%
Kleinhesselink, Andrew R. 2017. ``Direct and indirect effects of climate
change on plant populations and communities in sagebrush steppe.''
Dissertation, Utah State University.

\leavevmode\hypertarget{ref-Knapp2017}{}%
Knapp, Alan K, Meghan L Avolio, Claus Beier, Charles J W Carroll, Scott
L Collins, Jeffrey S Dukes, Lauchlan H Fraser, et al. 2017. ``Pushing
precipitation to the extremes in distributed experiments:
recommendations for simulating wet and dry years.'' \emph{Global Change
Biology} 23 (5):1774--82. \url{https://doi.org/10.1111/gcb.13504}.

\leavevmode\hypertarget{ref-Knapp2012}{}%
Knapp, Alan K, John M Briggs, and Melinda D Smith. 2012. ``Community
stability does not preclude ecosystem sensitivity to chronic resource
alteration.'' \emph{Functional Ecology} 26 (6):1231--3.
\url{https://doi.org/10.1111/j.1365-2435.2012.02053.x}.

\leavevmode\hypertarget{ref-Knapp2015}{}%
Knapp, Alan K., Charles J.W. Carroll, Elsie M. Denton, Kimberly J. La
Pierre, Scott L. Collins, and Melinda D. Smith. 2015. ``Differential
sensitivity to regional-scale drought in six central US grasslands.''
\emph{Oecologia} 177 (4):949--57.
\url{https://doi.org/10.1007/s00442-015-3233-6}.

\leavevmode\hypertarget{ref-Kulmatiski2017a}{}%
Kulmatiski, Andrew, Peter B. Adler, John M. Stark, and Andrew T.
Tredennick. 2017. ``Water and nitrogen uptake are better associated with
resource availability than root biomass.'' \emph{Ecosphere} 8 (3).
\url{https://doi.org/10.1002/ecs2.1738}.

\leavevmode\hypertarget{ref-LaPierre2016}{}%
La Pierre, Kimberly J., Dana M. Blumenthal, Cynthia S. Brown, Julia A.
Klein, and Melinda D. Smith. 2016. ``Drivers of Variation in Aboveground
Net Primary Productivity and Plant Community Composition Differ Across a
Broad Precipitation Gradient.'' \emph{Ecosystems} 19 (3):521--33.
\url{https://doi.org/10.1007/s10021-015-9949-7}.

\leavevmode\hypertarget{ref-Lemoine2016}{}%
Lemoine, Nathan P, Justin Sheffield, Jeffrey S Dukes, Alan K Knapp, and
Melinda D Smith. 2016. ``Terrestrial Precipitation Analysis (TPA): A
resource for characterizing long-term precipitation regimes and
extremes.'' \emph{Methods in Ecology and Evolution} 7 (11):1396--1401.
\url{https://doi.org/10.1111/2041-210X.12582}.

\leavevmode\hypertarget{ref-Oksanen2016}{}%
Oksanen, Jari. 2016. ``Vegan: ecological diversity.''
\url{https://doi.org/10.1029/2006JF000545}.

\leavevmode\hypertarget{ref-Peters2012}{}%
Peters, Debra P C, Jin Yao, Osvaldo E. Sala, and John P. Anderson. 2012.
``Directional climate change and potential reversal of desertification
in arid and semiarid ecosystems.'' \emph{Global Change Biology} 18
(1):151--63. \url{https://doi.org/10.1111/j.1365-2486.2011.02498.x}.

\leavevmode\hypertarget{ref-R2016}{}%
R Core Team. 2016. \emph{R: A Language and Environment for Statistical
Computing}. Vienna, Austria.

\leavevmode\hypertarget{ref-Sala1992a}{}%
Sala, O. E., W. K. Lauenroth, and W. J. Parton. 1992. ``Long-term soil
water dynamics in the shortgrass steppe.'' \emph{Ecology} 73
(4):1175--81. \url{https://doi.org/10.2307/1940667}.

\leavevmode\hypertarget{ref-Seger1987}{}%
Seger, J. 1987. ``What is bet-hedging?'' In \emph{Oxford Surveys in
Evolutionary Biology}, edited by P.H. Harvey and L. Partridge, 182--211.
Oxford: Oxford University Press.
\url{http://ci.nii.ac.jp/naid/10004816581/\%5Cnpapers2://publication/uuid/BE92F8E9-1B87-4E50-85C5-AA134A0EBAF0}.

\leavevmode\hypertarget{ref-Smith2011}{}%
Smith, M.D. 2011. ``An ecological perspective on extreme climatic
events: A synthetic definition and framework to guide future research.''
\emph{Journal of Ecology} 99 (3):656--63.
\url{https://doi.org/10.1111/j.1365-2745.2011.01798.x}.

\leavevmode\hypertarget{ref-Smith2009}{}%
Smith, Melinda D., Alan K. Knapp, and Scott L. Collins. 2009. ``A
framework for assessing ecosystem dynamics in response to chronic
resource alterations induced by global change.'' \emph{Ecology} 90
(12):3279--89. \url{https://doi.org/10.1890/08-1815.1}.

\leavevmode\hypertarget{ref-rstan2016}{}%
Stan Development Team. 2016a. ``Rstan: the R interface to Stan, Version
2.14.1.'' \url{http://mc-stan.org/rstan.html}.

\leavevmode\hypertarget{ref-stan2016}{}%
---------. 2016b. ``Stan: A C++ Library for Probability and Sampling,
Version 2.14.1.''

\leavevmode\hypertarget{ref-Wilcox2016}{}%
Wilcox, Kevin R, John M Blair, Melinda D Smith, and Alan K Knapp. 2016.
``Does ecosystem sensitivity to precipitation at the site-level conform
to regional-scale predictions?'' \emph{Ecology} 97 (3):561--68.
\url{https://doi.org/10.1890/15-1437}.

\leavevmode\hypertarget{ref-Wilcox2017}{}%
Wilcox, Kevin R., Zheng Shi, Laureano A. Gherardi, Nathan P. Lemoine,
Sally E. Koerner, David L. Hoover, Edward Bork, et al. 2017.
``Asymmetric responses of primary productivity to precipitation
extremes: A synthesis of grassland precipitation manipulation
experiments.'' \emph{Global Change Biology} 23 (10):4376--85.
\url{https://doi.org/10.1111/gcb.13706}.



\end{document}
