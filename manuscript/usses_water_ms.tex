%% Submissions for peer-review must enable line-numbering
%% using the lineno option in the \documentclass command.
%%
%% Preprints and camera-ready submissions do not need
%% line numbers, and should have this option removed.
%%
%% Please note that the line numbering option requires
%% version 1.1 or newer of the wlpeerj.cls file, and
%% the corresponding author info requires v1.2

\documentclass[fleqn,10pt,lineno]{wlpeerj} % for journal submissions

% ZNK -- Adding headers for pandoc

\setlength{\emergencystretch}{3em} 
\providecommand{\tightlist}{
\setlength{\itemsep}{0pt}\setlength{\parskip}{0pt}}
\usepackage{lipsum}
\usepackage[unicode=true]{hyperref}
\usepackage{longtable}



\usepackage{setspace} \usepackage{todonotes} \usepackage{rotating}
\usepackage{color, soul} \usepackage{sectsty} \usepackage{bm,mathrsfs}
\usepackage{lipsum}

\title{Ecosystem functional response across precipitation extremes in a
sagebrush steppe}

\author[1,*]{Andrew T. Tredennick}

\corrauthor[1,*]{Andrew T. Tredennick}{\href{mailto:atredenn@gmail.com}{\nolinkurl{atredenn@gmail.com}}}
\author[1,2,*]{Andrew R. Kleinhesselink}

\author[3]{J. Bret Taylor}

\author[1]{Peter B. Adler}


\affil[1]{Department of Wildland Resources and the Ecology Center, Utah State
University, Logan, Utah 84322}
\affil[2]{Department of Ecology and Evolutionary Biology, University of
California, Los Angeles, Los Angeles, California 90095}
\affil[3]{United States Department of Agriculture, Agricultural Research Service,
U.S. Sheep Experiment Station, Dubois, Idaho 83423}


%
% \author[1]{First Author}
% \author[2]{Second Author}
% \affil[1]{Address of first author}
% \affil[2]{Address of second author}
% \corrauthor[1]{First Author}{f.author@email.com}

% 
\begin{abstract}
\textbf{Background.} Precipitation is predicted to become more variable
in the western United States, meaning years of above and below average
precipitation will become more common. Periods of extreme precipitation
are major drivers of interannual variability in ecosystem functioning in
water limited communities, but how ecosystems respond to these extremes
over the long-term may shift with precipitation means and variances.
Long-term changes in ecosystem functional response could reflect
compensatory changes in species composition or species reaching
physiological thresholds at extreme precipitation levels.

\textbf{Methods.} We conducted a five year precipitation manipulation
experiment in a sagebrush steppe ecosystem in Idaho, United States. We
used drought and irrigation treatments (approximately 50\%
decrease/increase) to investigate whether ecosystem functional response
remains consistent under sustained high or low precipitation. We
recorded data on aboveground net primary productivity (ANPP), species
abundance, and soil moisture. We fit a generalized linear mixed effects
model to determine if the relationship between ANPP and soil moisture
differed among treatments. We used nonmetric multidimensional scaling to
quantify community composition over the five years.

\textbf{Results.} Ecosystem functional response, defined as the
relationship between soil moisture and ANPP was similar among drought
and control treatments, but the irrigation treatment had a lower slope
than the control treatment. However, all estimates for the effect of
soil moisture on ANPP overlapped zero, indicating the relationship is
weak and uncertain regardless of treatment. There was also large spatial
variation in ANPP within-years, which contributes to the uncertainty of
the soil moisture effect. Plant community composition was remarkably
stable over the course of the experiment and did not differ among
treatments.

\textbf{Discussion.} Despite some evidence that ecosystem functional
response became less sensitive under sustained wet conditions, the
response of ANPP to soil moisture was consistently weak and community
composition was stable. The similarity of ecosystem functional responses
across treatments was not related to compensatory shifts at the plant
community level, but instead may reflect the insensitivity of the
dominant species to soil moisture. These species may be successful
precisely because they have evolved life history strategies which buffer
them against precipitation variability.
% Dummy abstract text. Dummy abstract text. Dummy abstract text. Dummy abstract text. Dummy abstract text. Dummy abstract text. Dummy abstract text. Dummy abstract text. Dummy abstract text. Dummy abstract text. Dummy abstract text.
\end{abstract}

\begin{document}

\flushbottom
\maketitle
\thispagestyle{empty}

\definecolor{blue}{rgb}{0,0,0.7} \newcommand{\new}{\textcolor{blue}}

\newcommand\blfootnote[1]{%
  \begingroup
  \renewcommand\thefootnote{}\footnote{#1}%
  \addtocounter{footnote}{-1}%
  \endgroup
}

\blfootnote{*Authors contributed equally to this work.}

\reversemarginpar

\section{INTRODUCTION}\label{introduction}

\new{Water availability is a major driver of annual net primary productivity (ANPP) in grassland ecosystems}
\citep{Huxman2004, Hsu2012}.
\new{Therefore, projected changes in precipitation regimes, associated with global climate change, will impact grassland ecosystem functioning.
Increases in precipitation variability will result in prolonged periods of extremely wet or dry years, and predicting the consequences of such chronic resource alterations requires understanding how ecosystems respond to above and below average precipitation levels.}

At any given site, the relationship between ANPP and water availability
(e.g., soil moisture) can be characterized by regressing historical
observations of ANPP on observations of soil moisture. Fitted functional
responses can then be used to infer how ANPP may change under future
precipitation regimes \citep[e.g.,][]{Hsu2012}. A problem with this
approach is that it requires extrapolation if future precipitation falls
outside the historical range of variability
\citep{Smith2011, Peters2012}. For example, the soil moisture-ANPP
relationship may be linear within the historical range of interannual
variation, but could saturate at higher levels of soil moisture. In
fact, saturating relationships are actually common
\citep{Hsu2012, Gherardi2015a}, perhaps because other resources, like
nitrogen, become more limiting in wet years than dry years. Failure to
accurately estimate the curvature of the soil moisture-ANPP relationship
will lead to over- or underprediction of ANNP under extreme
precipitation \citep{Peters2012}.

Another problem with relying on historical ecosystem functional
responses to predict impacts of altered precipitation regimes is that
these relationships themselves might shift over the long-term. Shifts in
species identities and/or abundances can alter an ecosystem's functional
response to water availability because different species have different
physiological thresholds. \citet{Smith2009} introduced the `Hierarchical
Response Framework' for understanding the interplay of community
composition and ecosystem functioning in response to long-term shifts in
resources. In the near term, ecosystem functioning such as ANPP will
reflect the physiological responses of individual species to the
manipulated resource level. For example, ANPP may decline under
simulated drought because the initial community consisted of
drought-intolerant species \citep{Hoover2014}. Ecosystem functioning may
recover over longer time spans as new species colonize or initial
species reorder in relative abundance. It is also possible that
ecosystem functioning shifts to a new mean state, reflecting the suite
of species in the new community \citep{Knapp2012}.

Experimental manipulations of limiting resources, like precipitation,
offer a route to understanding how ecosystems will respond to resource
levels that fall outside the historical range of variability
\citep{Avolio2015, Gherardi2015, Knapp2017}. Altering the amount of
precipitation over many years should provide insight into the time
scales at which water-limited ecosystems respond to chronic resource
alteration. We propose four alternative scenarios for the effect of
precipitation manipulation on the ecosystem functional response to soil
moisture based on the Hierarchical Response Framework (Fig. 1). We
define `ecosystem functional response' as the relationship between
available soil moisture and ANPP. The four scenarios are based on
possible outcomes at the community (e.g., community composition) and
ecosystem (e.g., soil moisture-ANPP regression) levels.

First, altered precipitation might have no effect on either ecosystem
functional response or community composition (Fig. 1, top left). In this
case, changes in ANPP would be well predicted by the current, observed
soil moisture-ANPP relationship. This corresponds to the early phases of
the Hierarchical Response Framework, where ecosystem response follows
the physiological responses of individual species. Second, the ecosystem
functional response might change while community composition remains the
same (Fig. 1, top right). A saturating soil moisture-ANPP response fits
this scenario, where individual species hit physiological thresholds or
are limited by some other resource. Third, the ecosystem functional
response might be constant but community composition changes (Fig. 1,
bottom left). In this case, changes in species' identities and/or
abundances occur in response to altered precipitation levels and species
more suited to the new conditions compensate for reduced function of
initial species. Fourth, and last, both ecosystem functional response
and community composition could change (Fig. 1, bottom right). New
species, or newly abundant species, with different physiological
responses completely reshape the ecosystem functional response.

All four outcomes are possible in any given ecosystem, but the time
scales at which the different scenarios play out likely differ
\citep{Smith2009, Wilcox2016, Knapp2017}. To determine these time
scales, we need to amass information on how quickly ecosystem functional
responses change in different ecosystems. We also need to understand
whether changes at the ecosystem level are driven by community level
changes or individual level responses.

To that end, here we report the results of a five-year precipitation
manipulation experiment in a sagebrush steppe grassland. We imposed
drought and irrigation treatments (approximately \(\pm50\%\)) and
measured ecosystem (ANPP) and community (species composition) responses.
We focus on how the drought and irrigation treatments affect the
relationship between interannual variation in available soil moisture
and interannual variation in ANPP, and if community dynamics underlie
the ecosystem responses. In particular, we are interested in the
consistency of the soil moisture-ANPP relationship among treatments. Is
the relationship steeper under the drought treatment at low soil
moisture? Does the relationship saturate under the irrigation treatment
at high soil moisture? To answer these questions we fit a generalized
linear mixed effects model to test whether the regressions differed
among treatments. We also analyzed community composition and
\new{the sensitivity of ANPP to drought and irrigation treatments} over
time, allowing us to place our experimental results within the framework
our scenarios (Fig. 1).

\section{METHODS}\label{methods}

\subsection{Study Area}\label{study-area}

We conducted our precipitation manipulation experiment in a sagebrush
steppe community at the USDA, U.S. Sheep Experiment Station (USSES) near
Dubois, Idaho (44.2\(^{\circ}\) N, 112.1\(^{\circ}\) W), 1500 m above
sea level. The plant community is dominated by the shrub
\emph{Artemesia tripartita} and three perennial bunchgrasses,
\emph{Pseudoroegneria spicata}, \emph{Poa secunda}, and
\emph{Hesperostipa comata}. During the period of our experiment (2011 --
2015), average mean annual precipitation was 265 mm
year\(\phantom{}^{-1}\) and mean monthly temperature ranged from
-5.2\(^{\circ}\)C in January to 21.8\(^{\circ}\)C in July. Between 1926
and 1932, range scientists at the USSES established 26 permanent 1
m\(^2\) quadrats to track vegetation change over time. In 2007, we
relocated 14 of the original quadrats, six of which were inside a large,
permanent livestock exclosure. We use these six plots as control plots
(i.e.~ambient precipitation) in the experiment described below.

In spring 2011, we established 16 new 1 m\(^2\) plots located in the
same exclosure as the six control plots. We avoided areas on steep hill
slopes, areas with greater than 20\% cover of bare rock, and areas with
greater than 10\% cover of the shrubs \emph{Purshia tridentata} and/or
\emph{Amelanchier utahensis}. We established the new plots in pairs and
randomly assigned each plot in a pair to receive a ``drought'' or
``irrigation'' treatment.

\subsection{Precipitation Experiment}\label{precipitation-experiment}

Drought and irrigation treatments were designed to decrease and increase
the amount of ambient precipitation by 50\%. To achieve this, we used a
system of rain-out shelters and automatic irrigation
\citep{Gherardi2013}. The rain-out shelters consisted of transparent
acrylic shingles 1-1.5 m above the ground that covered an area of
\(2.5\times2\) m. The shingles intercepted approximately 50\% of
incoming rainfall, which was channeled into 75 liter containers.
Captured rainfall was then pumped out of the containers and sprayed on
to the adjacent irrigation plot via two suspended sprinklers. Pumping
was triggered by float switches once water levels reached about 20
liters. We disconnected the irrigation pumps each October and
reconnected them each April. The rain-out shelters remained in place
throughout the year.

We monitored soil moisture in four of the drought-irrigation pairs using
Decagon Devices (Pullman, Washington) 5TM and EC-5 soil moisture
sensors. We installed four sensors around the edges of each 1x1 m census
plot, two at 5 cm soil depth and two at 25 cm soil depth. We also
installed four sensors in areas nearby the four selected plot pairs to
measure ambient soil moisture at the same depths. Soil moisture
measurements were automatically logged every four hours. We coupled this
temporally intensive soil moisture sampling with spatially extensive
readings \new{taken with a handheld EC-5 sensor} at six points within
all 16 plots and associated ambient measurement areas. These snapshot
data were collected on 06-06-2012, 04-29-2015, 05-07-2015, 06-09-2015,
and 05-10-2016\footnote{Dates formatted as: mm-dd-yyyy.}.

Analyzing the response to experimental treatments was complicated by the
fact that we did not directly monitor soil moisture in each plot on each
day of the experiment. Only a subset of plots were equipped with soil
moisture sensors, and within those plots, one or more of the sensors
frequently failed to collect data. To remedy these problems, and to
produce average daily soil moisture values for the ambient, drought, and
irrigation treatments, we used a statistical model to predict the
average treatment effects on soil moisture during the course of the
experiment.

\new{We first averaged the observed soil moisture readings for each day ($d$) and plot ($i$), $x_{i,d}$.
Plots were organized in groups of three, with each group ($g$) containing consisting of a control, a drought, and an irrigation plot.
Within each group of plots we standardized the irrigation and drought effects on soil moisture relative to the ambient soil moisture conditions in the control plot. Specifically we subtracted the ambient daily soil moisture measured in the control plot from the soil moisture measured within the drought and irrigation plots within each group and then divided by the standard deviation of daily soil moisture values measured in the control plot ($\Delta x_{g,d,\text{trt.}} = (x_{g,d,\text{trt.}} - x_{g,d,\text{control}})/sd(x_{g,d,\text{control}})$) where $\Delta x$ is the standardized treatment effect and $x_{g,d,\text{trt.}}$ is the raw soil moisture measure for plot group $g$ on day $d$ and treatment $\text{trt.}$.
These transformations ensured that the treatment effects in each plot were appropriately scaled by the local ambient conditions within each plot group.}

\new{We then modeled these daily deviations ($\Delta x_\text{trt.}$)}
from ambient conditions using a linear mixed effects model with
independent variables for treatment (irrigation or drought), season
(winter, spring, summer, fall), rainfall, and all two-way interactions.
Rainy days were defined as any day in which precipitation was recorded
and average temperature was above 3\(^{\circ}\)C. The day immediately
following rainfall was also classified as rainy. We fit the model using
the \texttt{lme4::lmer()} function \citep{Bates2015} in R \citep{R2016},
with random effects for plot group and date. We weighted observations by
the number of unique sensors or spot measurements that were taken in
each plot on that day.
\new{We then used the model to predict the deviations from ambient conditions produced in the treated plots on each day of the experiment.  
We added these predicted deviations to the average daily ambient soil moisture to generate predictions for daily soil moisture in all of the treated plots: $\bar{x}_{d, \text{trt.}} = \Delta \bar{x}_{d,\text{trt.}} + \bar{x}_{d,\text{control}}$, where $\bar{x}_{d,\text{trt.}}$ is the average predicted daily soil moisture in the treated plots and $\bar{x}_{d,\text{control}}$ is the daily ambient soil moisture averaged across all control plots.}
We could only predict soil moisture in the treated plots on days for
which we took at least one ambient soil moisture measurement. This
procedure allowed us to predict daily soil moisture conditions for all
plots, even on those days when some of our direct treatment measurements
were missing due to malfunction of the sensors or dataloggers. See
\citet{Kleinhesselink2017b} for complete details on our approach to
estimating soil moisture.

\subsection{Data Collection}\label{data-collection}

We estimated aboveground net primary productivity (ANPP) using a
radiometer to relate ground reflectance to plant biomass \citep[see][
for a review]{Byrne2011}. We recorded ground reflectance at four
wavelengths, two associated with red reflectance (626 nm and 652 nm) and
two associated with near-infrared reflectance (875 nm and 859 nm). At
each plot in each year, we took four readings of ground reflectances at
the above wavelengths. We also took readings in 12 (2015), 15 (2012,
2013, 2014), or 16 (2016) calibration plots adjacent to the experimental
site, in which we harvested all aboveground biomass produced in the
current year (we excluded litter and standing dead material), dried it
to a constant weight at 60\(^{\circ}\)C, and weighed it to estimate
ANPP. We harvested at peak biomass each year, typically in late June.

For each plot and year, we averaged the four readings for each
wavelength and then calculated a greenness index based on the same bands
used to calculate NDVI using the MODIS (Moderate Resolution Imaging
Spectroradiometer) and AVHRR (Advanced Very High Resolution Radiometer)
bands for NDVI. We regressed the greenness index against the dry biomass
weight from the ten calibration plots to convert the greenness index to
ANPP. We fit regressions to a MODIS-based index and an AVHRR-based index
for each year and retained the regression with the better fit based on
\(R^2\) values. We then predicted ANPP using the best regression
equation for each year (Appendix 1).
\new{Our analysis, described below, can be done using NDVI directly, but we focus on ANPP to more easily relate our findings to previous work and because our results do not change if we analyze NDVI directly (Appendix 2).}

Species composition data came from two sources: yearly census maps for
each plot made using a pantograph \citep{Hill1920} and yearly counts of
annual species in each plot. From these sources, we determined the
density of all annuals and perennials forbs, the basal cover of
perennial grasses, and the canopy cover of shrubs. We made a large
plot-treatment-year by species matrix, where columns were filled with
either basal cover or density, depending on the measurement made for the
particular species. We standardized the values in each column so we
could directly compare species quantified with different metrics
(density, basal cover, and canopy cover). This puts all abundance values
on the same scale, meaning that common and rare species are weighted
equally. Assuming that rare species will respond to treatments more than
common ones, our approach is biased towards detecting compositional
changes.

\subsection{Data Analysis}\label{data-analysis}

\subsubsection{Ecosystem functional
response}\label{ecosystem-functional-response}

Our main goal was to test whether the relationship between ANPP and soil
moisture differed among the drought, control, and irrigation treatments.
Based on our own observations and previous work at our study site
\citep{Blaisdell1958, Dalgleish2011, Adler2012}, we chose to use
cumulative volumetric water content from March through June as our
metric of soil moisture (hereafter referred to as `VWC'). To achieve
this goal, we fit a generalized linear mixed effects regression model
with log(ANPP) as the response variable and VWC and treatment as fixed
effects. Plot and year of treatment were included as random effects to
account for non-independence of observations, as described below. We
log-transformed ANPP to account for heteroscedasticity. Both log(ANPP)
and VWC were standardized to have mean 0 and unit variance before
fitting the model {[}i.e., \((x_i - \bar{x})/\sigma_x\){]}.

Our model is defined as follows:

\vspace{-2em}

\begin{align}
\mu_{i} &= \boldsymbol{\beta}\textbf{x}_i + \boldsymbol{\gamma}_{j(i)}\textbf{z}_i + \eta_t, \\
\textbf{y} &\sim \text{Normal} \left(\boldsymbol{\mu}, \sigma^2 \right),
\end{align}

\noindent{}where \(\mu_{i}\) is the deterministic prediction from the
regression model for observation \(i\), which is associated with plot
\(j\) and treatment year \(t\). \(\boldsymbol{\beta}\) is the vector of
coefficients for the fixed effects in the design matrix \(\textbf{X}\).
Each row of the design matrix represents a single observation
(\(\textbf{x}_i\)) and is a vector with the following elements: 1 for
the intercept, a binary 0 or 1 if the treatment is ``drought'', a binary
0 or 1 if the treatment is ``irrigation'', the scaled value of VWC,
binary ``drought'' value times VWC, and binary ``irrigation'' value
times VWC. Thus, our model treats ``control'' observations as the main
treatment and then estimates intercept and slope offsets for the
``drought'' and ``irrigation'' treatments. We use our model to test two
statistical hypotheses based on the questions in our Introduction:

\begin{description}
\item [H1.] The coefficient for drought$\times$VWC is positive and different from zero.
\item [H2.] The coefficient for irrigation$\times$VWC is negative and different from zero.
\end{description}

\noindent{}These hypotheses are based on evidence that
precipitation-ANPP relationships often saturate with increasing
precipitation \citep{Hsu2012, Gherardi2015a}.

We include two random effects to account for the fact that observations
within plots and years are not independent. Specifically, we include
plot-specific offsets (\(\boldsymbol{\gamma}\)) for the intercept and
slope terms and year-specific intercept offsets (\(\eta_t\)). The
covariate vector \(\textbf{z}_i\) for each observation \(i\) has two
elements: a 1 for the intercept and the scaled value of VWC for that
plot and year. The plot-specific coefficients are modeled
hierarchically, where plot level coefficients are drawn from a
multivariate normal distribution with mean 0 and a variance-covariance
structure that allows the intercept and slope terms to be correlated:

\vspace{-1em}

\begin{align}
\boldsymbol{\gamma}_{j(i)} &\sim \text{MVN} \left( 0, \Sigma  \right),
\end{align}

\noindent{}where \(\Sigma\) is the variance-covariance matrix and
\(j(i)\) reads as ``plot \(j\) associated with observation \(i\)''. The
random year effects (\(\boldsymbol{\eta}\)) are drawn from a normal
prior with mean 0 and standard deviation \(\sigma_{\text{year}}\), which
was drawn from a half-Cauchy distribution.
\new{We used a vague prior distribution for each $\beta$: $\boldsymbol{\beta} \sim \text{Normal}(0,5)$.}
A full description of our model and associated R \citep{R2016} code is
in Appendix 2.

We fit the model using a Bayesian approach and obtained posterior
estimates of all unknowns via the No-U-Turn Hamiltonian Monte Carlo
sampler in Stan \citep{stan2016}. We used the R package `rstan'
\citep{rstan2016} to link R \citep{R2016} to Stan. We obtained samples
from the posterior distribution for all model parameters from four
parallel MCMC chains run for 10,000 iterations, saving every
\(10^{\text{th}}\) sample. Trace plots of all parameters were visually
inspected to ensure well-mixed chains and convergence. We also made sure
all scale reduction factors (\(\hat{R}\) values) were less than 1.1
\citep{Gelman2009}.

\subsubsection{Sensitivity of ANPP over
time}\label{sensitivity-of-anpp-over-time}

\new{Unfortunately, our data does not allow us to include the effect of time on ecosystem functional response in the model described above. 
This is because we lack sufficient within-year and within-treatment variation of soil moisture (i.e., within a year and treatment each plot shares the same value of VWC). 
Therefore, we conducted an independent analysis to determine if the sensitivity of ANPP to treatment changed over time.}

\new{We define `sensitivity' as $\frac{ \text{ANPP}_{\text{control}} - \text{ANPP}_{\text{treatment}} } { \text{VWC}_{\text{control}} - \text{VWC}_{\text{treatment}} }$, following}
\citet{Wilcox2017}.
\new{This metric of sensitivity measures the response of ANPP standardized to the magnitude of the treatment. Because our treatment and control plots are not paired, we conducted this analysis by comparing each treatment plot to the mean of the control plots in each year. Then, independently for each treatment, we regressed sensitivity against year of treatment using the \texttt{lm()} function in R}
\citep{R2016}.
\new{This analysis also allows us to link particularly sensitive treatment years to changes in community composition.}

\subsubsection{Community composition over
time}\label{community-composition-over-time}

We used nonmetric multidimensional scaling (NMDS) based on Bray-Curtis
distances to identify temporal changes in community composition among
treatments. We first calculated Bray-Curtis distances among all plots
for each year of the experiment and then extracted those distances for
use in the NMDS. Some values of standardized species' abundances were
negative, which is not allowed for calculating Bray-Curtis distances. We
simply added `2' to each abundance value to ensure all values were
greater than zero. We plotted the first two axes of NMDS scores to see
if community composition overlapped, or not, among treatments in each
year. We used the \texttt{vegan::metaMDS()} function \citep{Oksanen2016}
to calculate Bray-Curtis distances and then to run the NMDS analysis. We
used the \texttt{vegan::adonis()} function \citep{Oksanen2016} to
perform permutational multivariate analysis of variance to test whether
treatment plots formed distinct groupings. To test whether treatment
plots were equally dispersed, or not, we used the
\texttt{vegan::betadisper()} function \citep{Oksanen2016}.

\new{We conducted the above analysis for all species, and then conducted separate analyses for perennial species only and annual species only. Annual species have shorter life spans, so conducting separate analyses allows us to discover if community responses are restricted to short-lived species that might respond more quickly to altered precipitation regimes. Given the dominance of perennial species in our system, shifts in the annual plant community could be masked in the analysis of the full community.}

\subsubsection{Reproducibility}\label{reproducibility}

All R code and data necessary to reproduce our analysis has been
archived on Figshare (\emph{link here after acceptance}) and released on
GitHub (\url{https://github.com/atredennick/usses_water/releases/v0.1}).
We also include annotated Stan code in our model description in Appendix
2.

\section{RESULTS}\label{results}

Ambient precipitation was variable over the five years of the study
(Fig. 2A), meaning our experiment captured a range of environmental
conditions. ANPP varied from a minimum of 74.5 g m\(^{-2}\) in 2014 to a
maximum of 237.1 g m\(^{-2}\) in 2016 when averaged across treatments
(Fig. 2C). ANPP was slightly higher in irrigation plots
\new{(on average 17\% higher)} and slightly lower in drought plots
\new{(on average 23\% lower) relative to control plots} (Fig. 2C),
corresponding to estimated soil volumetric water content (VWC)
differences among treatments (Fig. 2B).
\new{VWC in drought plots was 12\% less than in control plots on average, and VWC in irrigation plots was 19\% higher than in control plots on average.}
Such differences in soil VWC indicate our treatment infrastructure was
successful. ANPP was highly variable across plots within years,
\new{as indicated by the large and overlapping standard deviations}
(Fig. 2C).

Cumulative March-June soil moisture had a weak positive effect on ANPP
(Table 1; Fig. 3B). The effect of soil moisture for each treatment is
associated with high uncertainty, however, with 95\% Bayesian credible
intervals that \new{broadly} overlap zero (Table 1). Although the
parameter estimates for the effect of soil moisture overlap zero, the
posterior distributions of the slopes all shrank and shifted to more
positive values relative to the prior distributions (Fig. A2-2), which
indicates the data did influence parameter estimates beyond the
information from the uninformative priors. Ecosystem functional response
was similar among treatments (Table 1; Fig. 3B), but there is evidence
that the \new{slopes for the drought and} irrigation treatments are less
than the slope for the control treatment. This evidence comes from
interpreting the posterior distributions of the slope offsets for the
treatments. \new{From these distributions,} we calculate a \new{91}\%
one-tailed probability that the estimate is less than zero for the
irrigation treatment
\new{and a 96\% one-tailed probability that the estimate is less than zero for the drought treatment}
(Fig. 3A, right panel).

\new{Sensitivity of ANPP to irrigation was constant over the course of our experiment (Fig. x). Sensivitity of ANPP to drought, however, grew over time (\emph{P} = 0.0002; Fig. x)}.

Community composition was similar among treatments. The multidimensional
space of community composition overlapped among treatments in all years
and was equally dispersed in all years (Table 2; Fig. 4). Community
composition was also remarkably stable over time, with no evidence of
divergence among treatments (Table 2; Fig. 4).
\new{Splitting the analysis apart for perennials and annuals led to similar results.
The perennial community was stable over time and consistently similar among treatments (Fig. Ax-x).
There is some evidence that the annual community changed in response to treatment in two years (Fig. Ax-x), but these responses appear to come from very rare species since our analysis weights them equally (Appendix X).}

\section{DISCUSSION}\label{discussion}

Ecosystem response to a new precipitation regime depends on the
physiological responses of constituent species and the rate at which
community composition shifts to favor species better able to take
advantage of, or cope with, new resource levels \citep{Smith2009}.
Previous work has shown that community compositional shifts can be both
rapid, on the order of years \citep{Hoover2014}, and slow, on order of
decades \citep{Knapp2012, Wilcox2016}. A lingering question is how the
time scales of ecosystem response and community change vary among
ecosystems. Precipitation manipulation experiments can help answer this
question, especially if they push water availability outside the
historical range of variability for long periods.

The results of our five year experiment in a sagebrush steppe
\new{conforms to one} of our four predictions. Ecosystem functional
response under chronic \new{drought and irrigation} was different from
the control treatment, but community composition remained unchanged
(Fig. 3A, Fig. 4), representing the top right scenario in Fig. 1. The
decrease in the slope of the VWC \(\times\) productivity relationship in
the irrigated plots is consistent with a weaker response of productivity
to additional water in this ecosystem. This suggests that if average
soil moisture were pushed consistently higher than currently observed
ambient conditions, there would be a weaker relationship between
precipitation and productivity in this system.

\new{The decrease in the slope of the VWC $\times$ productivity relationship in the droughted plots is puzzling because it is counter to our expectation that the slope would be higher than the control plots.
However, we believe the significance of the slope offsets has more to do with noise than signal.
This is because the full distributions of} the slopes of the VWC
\(\times\) productivity relationship, \new{not just the offsets}, were
similar among treatments (Table 1). We therefore conclude that ecosystem
functional response is consistent (\new{similar values}) and weak
(\new{all broadly overlapping zero}) across all precipitation
treatments.

The similarity of ecosystem functional response (Fig. 3) and community
composition (Fig. 4) among treatments is surprising because grasslands
generally, and sagebrush steppe specifically, are considered
water-limited systems. \new{For example,} \citet{Huxman2004} and
\citet{Knapp2015}
\new{showed that semi-arid sites are more sensitive to drought than mesic sites, and}
\citet{Wilcox2017}
\new{found that semi-arid sites are particularly sensitive to irrigation treatments.}
\new{Based on these findings}, we expected ecosystem functional
response, community composition, or both to change under our treatments.
Why did our treatments fail to induce ecosystem or community responses?
We can think of three reasons. Two are limitations of our study, and one
is the life history traits of the species in our focal communities. We
first discuss the potential limitations of our study, and then discuss
the biological explanation.

First, it could be that our precipitation manipulations were not large
enough to induce a response. That is, a 50\% decrease in any given year
may not be abnormal given our site's historical range of variability
\citep{Knapp2017}. We cannot definitively rule out this possibility, but
we have reason to believe our treatments \emph{should} have been large
enough. Using the methods described by \citet{Lemoine2016}, we
calculated the percent reduction and increase of mean growing season
precipitation necessary to reach the 1\% and 99\% extremes of the
historical precipitation regime at our site (Fig. A4-1). The 1\%
quantile of precipitation at our site is 110 mm, a 47\% reduction from
the mean, and the 99\% quantile is 414 mm, a 77\% increase from mean
growing season precipitation (Appendix 4). Thus, our drought treatment
represented extreme precipitation amounts, especially in years where
ambient precipitation was below average (Fig. 2A). The irrigation
treatment may not qualify as extreme, yet that is the treatment where we
did observe an effect (Fig. 3A).

Second, ANPP at our site may be influenced by additional factors, not
only the cumulative March-June soil moisture covariate we included in
our statistical model. For example, temperature can impact ANPP directly
\citep{Epstein1997} and by exacerbating the effects of soil moisture
\citep{DeBoeck2011}. Measurements of soil moisture likely contain a
signal of temperature, through its impact on evaporation and
infiltration, but the measurements will not capture the direct effect of
temperature on metabolic and physiological processes. We also did not
redistribute snow across our treatments in the winter, and snow melt may
spur early spring growth. Failure to account for potentially important
covariates could explain the within-year spread of ANPP (Fig 2C, Fig.
3B) and the resulting uncertain relationship we observed between soil
moisture and ANPP across all treatments (Table 1, Fig. A2-2).
Unfortunately, we did not collect data on additional factors because our
focus was on precipitation.

Third, the life history traits of the dominant species in our study
ecosystem may explain the consistently positive, but weak and uncertain,
effect of soil moisture on ANPP (Table 1, Fig. 3). Species that live in
variable environments, such as cold deserts, must have strategies to
ensure long-term success as conditions vary. One strategy is bet
hedging, where species forego short-term gains to reduce the variance of
long-term success \citep{Seger1987}. In other words, species follow the
same conservative strategy every year, designed to minimize response to
environmental conditions. The dry and variable environment of the
sagebrush steppe has likely selected for bet hedging species that can
maintain function at low water availability and have weak responses to
high water availability. In so doing, the dominant species in our
ecosystem avoid ``boom and bust'' cycles, which corresponds to the weak
effect of soil moisture on ANPP (i.e., the Bayesian credible intervals
for the slopes overlapping zero; Table 1).

Another strategy to deal with variable environmental conditions is
avoidance, which would also result in a consistent ecosystem functional
response between drought and control treatments. For example, many of
the perennial grasses in our focal ecosystem avoid drought stress by
growing early in the growing season \citep[A.R. Kleinhesselink, personal
observation]{Blaisdell1958}. Furthermore, the dominant shrub in our
focal ecosystem, \emph{Artemisia tripartita}, has access to water deep
in the soil profile thanks to a deep root system
\citep{Kulmatiski2017a}.
\new{The dominance of our site by the shrub \emph{A. tripartita} may explain why our results do not conform to broader patterns of grassland sensitivity to precipitation manipulations}
\citep[e.g.,][]{Huxman2004, Knapp2015, Wilcox2017}.

\new{We did find that ANPP became more sensitive to drought over time (Fig. x).
Thus, the difference between ANPP, standardize by the difference in soil moisture, in drought and control plots grew over time.
The latter years of our experiment were relatively wet, meaning that the drought treatment could have the largest absolute reduction of soil moisture in those years.
But our metric of sensitivity accounts for differences among years because ANPP differences is standardized by the difference in soil moisture within the year.
Therefore, it is likely that the increased sensitivity to drought over time is due to the cimulative impact of drought on the dominant species.
Ecosystem functional response may soon shift if sensitivity continues to increase.}

\section{CONCLUSIONS}\label{conclusions}

Our results suggest the species in our focal plant community are
insensitive to to changes in precipitation regime,
\new{at least over the five years of our experiment}. Such insensitivity
can buffer species against precipitation variability in this semi-arid
ecosystem, making them successful in the long run. Longer, chronic
precipitation alteration might reveal plant community shifts that we did
not observe \citep[e.g.,][]{Wilcox2016}. For example, a long-term
increase in water availability could allow species that do not bet hedge
to gain prominence and dominate the ecosystem functional response.
\new{Time \emph{per se} may be especially relevant in our focal ecosystem where the perennial species are long-lived, meaning compositional turnover may take many more years than we report on here.}
Our results suggest compositional shifts would have the largest impact
at high rainfall because the current community maintained consistent
ecosystem functional response at very low water availability.
\new{However, the increased sensitivity of ANPP to drought conditions suggests a shift in ecosystem functional response may be near, and simply take more time than we observed here.}

\section{ACKNOWLEDGEMENTS}\label{acknowledgements}

We thank the many summer research technicians who collected the data
reported in this paper and the USDA, U.S. Sheep Experiment Station
(Dubois, ID) for facilitating work on their property. We also thank
Susan Durham for clarifying our thinking on the statistical analyses and
Kevin Wilcox for helpful discussions on analyzing community composition
data.
\new{Two anonymous reviewers and Elsie Denton provided thoughtful reviews that improved our paper.}

\section{FUNDING}\label{funding}

This research was supported by the Utah Agricultural Experiment Station,
Utah State University, and approved as journal paper number 9035. The
research was also supported by the National Science Foundation, through
a Postdoctoral Research Fellowship in Biology and Mathematics to ATT
(DBI-1400370), a Graduate Research Fellowship to ARK, and grants
DEB-1353078 and DEB-1054040 to PBA.

\section{AUTHOR CONTRIBUTIONS}\label{author-contributions}

\begin{itemize}
  \item Andrew T. Tredennick collected data, analyzed the data, wrote the paper, prepared figures and/or tables, reviewed drafts of the paper.
  \item Andrew R. Kleinhesselink conceived and designed the experiments, performed the experiments, collected data, analyzed the data, reviewed drafts of the paper.
  \item J. Bret Taylor contributed reagents/materials/analysis tools, reviewed drafts of the paper.
  \item Peter B. Adler conceived and designed the experiments, performed the experiments, collected data, analyzed the data, reviewed drafts of the paper.
\end{itemize}

\section{SUPPLEMENTAL INFORMATION}\label{supplemental-information}

\begin{description}
\item [Appendix 1.] Additional methods and information on estimating aboveground net primary productivity.
\item [Appendix 2.] \new{Results from analysis of NDVI without conversion to ANPP, Fig. A3-1.}
\item [Appendix 3.] Details of the hierarchical Bayesian model, Fig. A2-1, Fig. A2-2, and Fig. A2-3.
\item [Appendix 4.] Details on analysis of precipitation historical range of variability and Fig. A4-1.
\end{description}

\newpage{}

\section{TABLES}\label{tables}

\begin{table}[ht]
\centering
\caption{Summary statistics from the posterior distributions of coefficients for each treatment.} 
\begingroup\normalsize
\begin{tabular}{llrrrr}
  \hline
Coefficient & Treatment & Posterior Mean & Posterior Median & Lower 95\% BCI & Upper 95\% BCI \\ 
  \hline
Intercept & Control & 0.16 & 0.14 & -1.22 & 1.69 \\ 
  Intercept & Drought & 0.13 & 0.05 & -1.69 & 2.39 \\ 
  Intercept & Irrigation & -0.24 & -0.18 & -2.11 & 1.53 \\ 
  Slope & Control & 0.94 & 0.86 & -0.51 & 2.69 \\ 
  Slope & Drought & 0.65 & 0.57 & -0.99 & 2.62 \\ 
  Slope & Irrigation & 0.72 & 0.65 & -0.54 & 2.26 \\ 
   \hline
\end{tabular}
\endgroup
\end{table}\begin{table}[ht]
\centering
\caption{Results from statistical tests for clustering and dispersion of community composition among precipitation treatments. `adonis' tests whether treatments form unique clusters in multidimensial space; `betadisper' tests whether treatments have similar dispersion. For both tests, \emph{P} values greater than 0.05 indicate there is no support that the treatments differ.} 
\begingroup\normalsize
\begin{tabular}{rlrrrr}
  \hline
Year & Test & n & d.f. & \emph{F} & \emph{P} \\ 
  \hline
2011 & adonis &  21 &   2 & 1.02 & 0.45 \\ 
  2011 & betadisper &  21 &   2 & 2.22 & 0.14 \\ 
  2012 & adonis &  22 &   2 & 1.09 & 0.34 \\ 
  2012 & betadisper &  22 &   2 & 0.21 & 0.81 \\ 
  2013 & adonis &  22 &   2 & 1.25 & 0.13 \\ 
  2013 & betadisper &  22 &   2 & 0.44 & 0.65 \\ 
  2014 & adonis &  22 &   2 & 0.95 & 0.56 \\ 
  2014 & betadisper &  22 &   2 & 0.35 & 0.71 \\ 
  2015 & adonis &  21 &   2 & 1.05 & 0.39 \\ 
  2015 & betadisper &  21 &   2 & 3.01 & 0.07 \\ 
  2016 & adonis &  21 &   2 & 1.07 & 0.36 \\ 
  2016 & betadisper &  21 &   2 & 0.49 & 0.62 \\ 
   \hline
\end{tabular}
\endgroup
\end{table}

\newpage{}

\section{FIGURES}\label{figures}

\begin{figure}[!ht]
  \centering
      \includegraphics[width=4in]{../figures/hypothesis_figtable.png}
  \caption{Possible outcomes of chronic resource alteration based on the 'Hierarchical Response Framework' (Smith et al. 2009).}
\end{figure}

\newpage{}

\begin{figure}[!ht]
  \centering
      \includegraphics[height=7in]{../figures/data_panels.png}
  \caption{(A) Probability density of historical precipitation from 1926-2016, with the years of the experiment shown with arrows on the \emph{x}-axis. (B) \new{Estimated daily} soil volumetric water content (VWC) in each of the three treatments during the course of the experiment. (C) Mean (points) ANPP and its standard deviation (error bars) for each year of the experiment. Colors in panels B and C identify the treatment, as specified in the legend of panel C.}
\end{figure}

\newpage{}

\begin{figure}[!ht]
  \centering
      \includegraphics[width=5in]{../figures/glmm_main_results.png}
  \caption{Results from the generalized linear mixed effects model \new{(A-B) and the sensitivity analysis (C)}. (A) Posterior distributions for the effects of drought and irrigation on the intercept and slope of the productivity-soil moisture relationship. Treatment effects show the difference between the coefficients estimated in the treated plots and the control plots. Probabilities (``Pr $\lessgtr$ 0 ='') for each coefficient indicate the one-tailed probability that the coefficient is less than or greater than zero, depending on whether the median of the distribution is less than or greater than zero. The posterior densities were smoothed for visual clarity by increasing kernel bandwidth by a factor of five. (B) Scatterplot of the data and model estimates shown as solid lines. Model estimates come from treatment level coefficients (colored lines). Note that we show log(ANPP) on the \emph{y}-axis of panel B; this same plot can be seen on the arithmetic scale in supporting material Fig. A2-1. \new{(C) Regression of sensitivity against time for each treatment. Each point represents the sensitivity of ANPP in a plot relative to the mean of the control plots in that year. Only the significant regression for the drought treatment is plotted (\emph{P} = 0.0002).}}
\end{figure}

\newpage{}

\begin{figure}[!ht]
  \centering
      \includegraphics[width=5in]{../figures/sppcomp_bray_all.png}
  \caption{Nonmetric multidimensional scaling scores representing plant communities in each plot, colored by treatment.}
\end{figure}

\newpage{}


\bibliography{/Users/atredenn/Dropbox/Bibliography/usses_water.bib}


\end{document}
