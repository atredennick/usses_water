%% Submissions for peer-review must enable line-numbering
%% using the lineno option in the \documentclass command.
%%
%% Preprints and camera-ready submissions do not need
%% line numbers, and should have this option removed.
%%
%% Please note that the line numbering option requires
%% version 1.1 or newer of the wlpeerj.cls file, and
%% the corresponding author info requires v1.2

\documentclass[fleqn,10pt,lineno]{wlpeerj} % for journal submissions

% ZNK -- Adding headers for pandoc

\setlength{\emergencystretch}{3em} 
\providecommand{\tightlist}{
\setlength{\itemsep}{0pt}\setlength{\parskip}{0pt}}
\usepackage{lipsum}
\usepackage[unicode=true]{hyperref}
\usepackage{longtable}



\usepackage{lipsum} \usepackage{setspace} \usepackage{todonotes}
\usepackage{rotating} \usepackage{color, soul} \usepackage{sectsty}
\usepackage{bm,mathrsfs}

\title{Consistent ecosystem functional response during five years of drought
and deluge in a sagebrush steppe}

\author[1]{Andrew T. Tredennick}

\corrauthor[1]{Andrew T. Tredennick}{\href{mailto:atredenn@gmail.com}{\nolinkurl{atredenn@gmail.com}}}
\author[2]{Andrew R. Kleinhesselink}

\author[1]{Peter B. Adler}


\affil[1]{Department of Wildland Resources and the Ecology Center, Utah State
University, Logan, Utah 84322}
\affil[2]{Department of Ecology and Evolutionary Biology, University of
California, Los Angeles, Los Angeles, California 90095}


%
% \author[1]{First Author}
% \author[2]{Second Author}
% \affil[1]{Address of first author}
% \affil[2]{Address of second author}
% \corrauthor[1]{First Author}{f.author@email.com}

% 
\begin{abstract}
Precipitation is predicted to become more variable in the western U.S.,
meaning years of above and below average precipitation will become more
common. Such periods of drought and deluge could become major drivers of
plant community dynamics and ecosystem functioning in water limited
grasslands. Here we report the results of a five-year experiment where
we used drought and irrigation treatments (50\% decrease/increase) to
see how a sagebrush steppe plant community in Idaho will respond to
future climate changes. Ecosystem functional response, defined as the
relationship between soil moisture and aboveground net primary
productivity, was suprisingly consistent across drought, control, and
irrigation treatments. Aboveground net primary productivity (ANPP)
responded positively to availabile soil moisture, but the response did
not vary across treatments. There was also no evidence that treatment
effects grew over time. The similarity of ecosystem functional response
across treatment was not due to compensatory shifts at the plant
community level, where species composition among treatments was similar
and remarkably stable over the five years. At least in the short-term,
ecosystem functional response and community composition in this
sagebrush steppe system is resistant to increases and decreases in
growing season precipitation.
% Dummy abstract text. Dummy abstract text. Dummy abstract text. Dummy abstract text. Dummy abstract text. Dummy abstract text. Dummy abstract text. Dummy abstract text. Dummy abstract text. Dummy abstract text. Dummy abstract text.
\end{abstract}

\begin{document}

\flushbottom
\maketitle
\thispagestyle{empty}

\definecolor{blue}{rgb}{0,0,0.7} \definecolor{red}{rgb}{0.7,0,0}
\newcommand{\pba}{\textcolor{blue}} \newcommand{\ark}{\textcolor{red}}

\section{INTRODUCTION}\label{introduction}

Water availability is a strong driver of aboveground net primary
productivity (ANPP) in grassland ecosystems. Mean annual precipitation
(MAP) is strongly correlated with ANPP across sites \citep{Huxman2004},
and, in water-limited ecosystems, interannual variation in MAP is also
correlated with interannual variation in ANPP \citep{Hsu2012}. These
relationships suggest that chronic alterations in levels of
precipitation, associated with global climate change, will impact ANPP.

At a given site, the functional response of ANPP to water availability
(e.g., soil moisture) can be characterized by fitting a model to
historical observations of ANPP and soil moisture. However, the fitted
functional response may provide an incomplete picture because future
conditions are likely to be outside the historical range of variability
\citep{Smith2011}. For example, historical trends may show a linear soil
moisture-ANPP relationship, but that relationship could potentially
saturate if soil moisture is pushed far beyond typical levels.
Saturating relationships are actually common \citep{Hsu2012}, perhaps
because other resources, like nitrogen, become more limiting in wet
years than dry years. Knowing the curvature of the soil moisture-ANPP
relationship is critical for understanding how ecosystems will respond
to chronic alterations in water availability.

Another problem with relying on historical ecoystem functional responses
is that they are not static. Changes in species identities and
abundances can alter an ecosystem's functional response to water
availability because different species have different physiological
thresholds for producting biomass. \citet{Smith2009} introduced the
`Hierarchical Response Framework' (HRF) for understanding the interplay
of community composition and ecosystem functioning in response to
resource manipulations over time. In the near term, ecosystem
functioning such as annual net primary productivity (ANPP) will reflect
the physiological responses of individual species to the manipulated
resource level. For example, ANPP may decline under simulated drought
because the initial community consisted of drought-intolerant species
\citep{Hoover2014}. Over longer time spans, ecosystem functioning may
recover as new species colonize or initial species reorder in relative
abundance. For example, ANPP may initially decline, but eventually rise
back to pre-treatment levels once drought-tolerant species become more
abundant and compensate for drought-intolerant species
\citep{Hoover2014}. It is also possible that ecosystem functioning
shifts to a new mean state, reflecting the suite of species in the new
community \citep{Knapp2012}.

Manipulating potentially limiting resources offers a route to
understanding how ecosystems will respond to resource levels that fall
outside the historical range of variability
\citep{Avolio2015, Gherardi2015, Knapp2017}. Altering the amount of
precipitation over many years should provide insight into the time
scales at which water-limited ecosystems respond to chronic resource
alteration. Following the HRF, we propose four alternative predictions
for the effect of precipitation manipulation on the ecosystem functional
response to soil moisture, that is, the soil moisture-ANPP relationship
(Figure 1). The four predictions are based on possible outcomes at the
community (e.g., community composition) and ecosystem (e.g., soil
moisture-ANPP regression) levels.

First, altered precipitation changes neither ecosystem functional
response nor community composition (Fig. 1, top left). In this case,
changes in ANPP simply follow the soil moisture-ANPP relationship under
ambient conditions. This corresponds to the early phases of the HRF,
where ecosystem response is due to the physiological responses of
individual species. Second, the ecosystem functional response changes
but community composition remains the same (Fig. 1, top right). A
saturating soil moisture-ANPP response fits this scenario, where
individual species hit physiological thresholds or are limited by some
other resource. Third, the ecosystem functional response is consistent
but underlying community composition changes (Fig. 1, bottom left). In
this case, changes in species' indentities or abundances occur in
response to altered precipitation levels and species more suited to the
new conditions compensate for reduced function of initial species.
Fourth, and last, both ecosystem functional response and community
composition change (Fig. 1, bottom right). New species, or newly
abundant species, with different physiological responses completely
reshape the ecosystem functional response.

All four outcomes are possible in any given ecosystem, but the time
scales at which the different scenarios play out likely differ
\citep{Smith2009, Wilcox2016, Knapp2017}. Thus, our task is not to test
the validity of the HRF, but rather to amass information on how quickly
ecosystem functional responses change in different ecosystems. Likewise,
we need to understand whether changes at the ecosystem level are driven
community level changes or individual level responses.

To that end, here we report the results of a five-year precipitation
manipulation experiment in a sagebrush steppe grassland. We imposed
drought and irrigation treatments (50\% decrease/increase) and measured
ecosystem (ANPP) and community (species composition) responses. We focus
on how the drought and irrigation treatments affect the relationship
between available soil moisture and ANPP, and if community dynamics
underly the ecosystem responses. In particular, we are interested in the
consistency of the soil moisture-ANPP relationship among treatments. Is
the relationship steeper under the drought treatment, at low soil
moisture? Does the relationship saturate under the irrigation treatment,
at high soil moisture? To answer these questions we fit a random
intercepts, random slopes model to test whether the regressions differed
among treatments. We also analyzed community composition over time,
allowing us to place our experimental results within the framework of
the HRF and our competing predictions (Fig. 1).

\section{METHODS}\label{methods}

\subsection{Study Area}\label{study-area}

We conducted our precipitation manipulation experiment at the United
States Sheep Experimental Station (USSES) near Dubois, Idaho
(44.2\(^{\circ}\) N, 112.1\(^{\circ}\) W), 1500 m above sea level. The
vegetation is typical of high elevation sagebrush steppe. The plant
community is dominated by the shrub \emph{Artemesia tripartita} and
three perennial bunchgrasses, \emph{Pseudoroegneria spicata}, \emph{Poa
secunda}, and \emph{Hesperostipa comata}. During the period of our
experiment (2011 -- 2015), average mean annual precipitation was 265 mm
year\(\phantom{}^{-1}\) and mean monthly temperature ranged from
-5.2\(^{\circ}\)C in January to 21.8\(^{\circ}\)C in July.

\subsection{Precipitation Experiment}\label{precipitation-experiment}

Between 1926 and 1932, range scientists at the USSES established 26
permanent 1 m\(^2\) quadrats to track vegetation change over time. In
2007, we (well, one of us {[}P. Adler{]}) relocated 14 of the original
quadrats, six of which were inside a large, permanent livestock
exclosure. We use these six plots as control plots that have recieved no
treatment, just ambient precipitation. In spring 2011, we (well, two of
us {[}A. Kleinhesselink and P. Adler{]}) established 16 new 1 m\(^2\)
plots. We avoided areas on steep hill slopes, areas with greater than
20\% cover of bare rock, and areas with greater than 10\% cover of the
shrubs \emph{Purshia tridentata} and/or \emph{Amelanchier utahensis}. We
established the new plots in pairs and randomly assigned each plot in a
pair to receive a ``drought'' or ``irrigation'' treatment.

Drought and irrigation treatments were designed to decrease and increase
the amount of ambient precipitation by 50\%, respectively. To achieve
this, we used a system of rain-out shelters and automatic irrigation
\citep{Gherardi2013}. The rain-out shelters consisted of transparent
acrylic shingles 1-1.5 m above the ground that covered an area of
\(2.5\times2\) m. The shingles intercepted approximately 50\% of
incoming rainfall, which was channeled into 75 liter containers.
Captured rainfall was then pumped out of the containers and sprayed on
to the adjacent irrigation plot via two suspended sprinklers. Pumping
was triggered by float switches once water levels reached about 20
liters. We disconnected the irrigation pumps each fall and reconnected
them, often with difficulty, each spring. The rain-out shelters remained
in place throughout the year.

To make sure the treatments were having the desired effects, we
monitored soil moisture in four of the drought-irrigation pairs using
Decagon Devices (Pulman, Washington) 5TM and EC-5 soil moisture sensors.
We installed four sensors in each plot, two at 5 cm soil depth and two
at 25 cm soil depth. We also installed four sensors in areas nearby the
four selected plot pairs to measure ambient soil moisture at the same
depths. Soil moisture measurements were automatically logged every four
hours. We coupled this temporally intensive soil moisture sampling with
spatially extensive readings taken at six points within all 16 plots and
associated ambient measurement areas. These snapshot data were collected
on 06/06/2012, 04/29/2015, 05/07/2015, 06/09/2015, and 05/10/2016 using
a handheld EC-5 sensor.

\ark{ANDY: add paragraph on the statistical modeling of soil moisture to produce Figure 2B (same as your Figure 2 in the precip-demography-prediction dissertation chapter).}

\subsection{Data Collection}\label{data-collection}

We estimated aboveground net primary productivity (ANPP) using a
radiometer to relate ground reflectance to plant biomass \citep[see][
for a review]{Byrne2011}. We recorded ground reflectance at four
wavelengths, two associated with red reflectance (626 nm and 652 nm) and
two associated with near-infrared reflectance (875 nm and 859 nm). At
each plot in each year, we took four readings of ground reflectances at
the above wavelengths. We also took readings in ten calibration plots
adjacent to the experimental site, in which we harvested all aboveground
biomass, dried it to a constant weight at 60\(^{\circ}\)C, and weighed
it to estimate ANPP.

For each plot and year, we averaged the four readings for each
wavelength and then calculated NDVI using the MODIS and AVHRR
algorithms. To convert NDVI to ANPP we regressed NDVI against the dry
biomass weight from the ten calibration plots. We fit regressions to
MODIS-based NDVI and AVHRR-based NDVI for each year and retained the
regression with the better fit. Using the best regression equation for
each year, we predicted ANPP (Appendix 1).

Species composition data came from two sources: yearly census maps for
each plot made using a pantograph \citep{Hill1920} and yearly counts of
annual species in each plot. The maps record the spatial location of all
individuals in the plot and the basal cover of each individual with
cover greater than 1 cm. Using those annual maps, we aggregated over
individuals to calculate total basal cover for each species in each
plot. We made a large plot-treatment-year by species matrix, where
columns were filled with either basal cover or density, depending on the
measurement made for the particular species. So we could analyze the
different types of data together, we standardized the values in each
column. This puts all abundance values on the same scale, but comes with
the limitation that all species are weighted equally. Nonetheless, this
scaling approach allows a comprehensive view of community composition
dynamics through time.

\subsection{Data Analysis}\label{data-analysis}

Our main goal was to test whether the relationship between ANPP and
growing season precipitation (hereafter, precipiation) differed among
the drought, control, and irrigation treatments. To achieve this goal,
we fit a multi-level random intercept and random slope regression with
log(ANPP) as the response variabile and soil moisture (volumetric water
content) as the sole predictor. We fit the model under a Bayesian
framework, allowing us to test for treatment differences by comparing
the posterior distributions of the treatment-level coefficients
\citep[e.g.,][]{Tredennick2013}. Both log(ANPP) and soil moisture were
standardized to have mean 0 and unit variance before fitting the model
{[}i.e., \((x_i - \bar{x})/\sigma_x\){]}.

Our multi-level model has three grouping levels for coefficients,
representing the nested structure of the data: (i) overall coefficients,
(ii) treatment coefficients, and (iii) plot coefficients. Each
subsequent level is drawn from the distribution of coefficients at the
previous level. Formally, our model is defined as follows:

\vspace{-2em}

\begin{align}
\mu_{i(j(k(t)))} &= \beta_{0,j(k)} + \beta_{1,j(k)}x_i + \gamma_t, \\
y_{i(j(k(t)))} &\sim \text{Normal} \left(\mu_{i(j(k(t)))}, \sigma^2_{k} \right),
\end{align}

\noindent{}where \(\mu_{i(j(k))}\) is the deterministic prediction from
the regression model for observation \emph{i} for plot \emph{j}
associated with treatment \emph{k} in year \emph{t}, \(\beta_{0,j(k)}\)
is the intercept for plot \emph{j} associated with treatment \emph{k},
\(\beta_{1,j(k)}\) is the slope term for the effect of soil moisture for
plot \emph{j} associated with treatment \emph{k}, \(\gamma_t\) is the
intercept offset for year \emph{t}, and \(\sigma^2_k\) is the process
variance for treatment \emph{k}. Data include the standardized log(ANPP)
observations \(\left( y_{i(j(k(t)))} \right)\) and soil moisture
(\(x_i\)). Although we include observation subscript \emph{i} on the
\emph{x}s, observations within a treatment-year all share the same soil
moisture values.

The intercept and slope terms are modeled hierarchichally to account for
the non-independence of observations across years within plots and to
allow us to test the hypothesis that our treatments alter the ANPP-soil
moisture relationship. As noted above, plot-level coefficients are drawn
from treatment-level coefficients, which are drawn from overall
coefficients. We also include a covariance structure among the intercept
and slope at each level. Formally, our hierarhical structure is as
follows, where we drop the intercept (0) and slope (1) subscripts and
instead refer to a vector of coefficients, \(\boldsymbol{\beta}\):

\vspace{-1em}

\begin{align}
\boldsymbol{\beta}_{j(k)} &\sim \text{MVN} \left( \boldsymbol{\beta}_k, \boldsymbol{\Sigma}(k) \right), \\
\boldsymbol{\beta}_{k} &\sim \text{MVN} \left( \boldsymbol{\beta}, \boldsymbol{\Sigma}  \right), \\
\boldsymbol{\beta} &\sim \text{Normal} \left( 0, 1 \right),
\end{align}

\noindent{}where \(\boldsymbol{\beta}_{j(k)}\) is the vector of
regression coefficients (intercept and slope) for plot \emph{j}
associated with treatment \emph{k}, \(\boldsymbol{\beta}_{k}\) is the
vector of coefficients for each treatment, and \(\boldsymbol{\beta}\) is
the vector of overall coefficients. The plot- and treatment-level
coefficients are drawn from multivariate normal distributions with
covariance matrix \(\boldsymbol{\Sigma}\). For the plot-level
coefficients, each treatment has its own variance-covariance matrix
(i.e., \(\boldsymbol{\Sigma}(k)\)). The overall coefficients are drawn
from a normal prior with mean 0 and standard deviation 1. The random
year effects (\(\boldsymbol{\gamma}\)) are drawn from a normal prior
with mean 0 and standard deviation \(\sigma_{\text{year}}\), which was
drawn from a weibull distribution. A full description of model is in
Appendix 2.

We fit the model using a Bayesian approach, obtaining posterior
estimates of all unkowns via the No-U-Turn Hamiltonian Monte Carlo
sampler in Stan \citep{stan2016}. We used the R package `rstan'
\citep{rstan2016} to link R \citep{R2016} to Stan. We obtained samples
from the posterior distribution for all model parameters from four
parallel MCMC chains run for 10,000 iterations, saving every
\(10^{\text{th}}\) sample. Traceplots of all parameters were visually
inspected to ensure well-mixed chains and convergence. We also made sure
all scale reduction factors (\(\hat{R}\)) were less than 1.1.

To see if community composition differed among treatments through time,
we used non-dimensional multivariate scaling (NMDS) based on Bray-Curtis
distances. For each year of the experiment, we first calculated
Bray-Curtis distances among all plots, and then extracted those
distances for use in the NMDS. Because we standardized species'
abundances, some values were negative, which is not allowed for
calculating Bray-Curtis distances. We simply added `2' to each abundance
value to ensure all values were greater than zero. We plotted the first
two axes of NMDS scores to see if community composition overlapped, or
not, among treatments in each year. We used the `metaMDS()' function in
the R package `vegan' \citep{Oksanen2016} to calculate Bray-Curtis
distances and then to run the NMDS analysis. We used the
`vegan::adonis()' function \citep{Oksanen2016} to perform permutational
multivariate analysis of variance to test whether treatment plots formed
distinct groupings. To test whether treatment plots were equally
dispersed, or not, we used the `vegan::betadisper' function
\citep{Oksanen2016}.

All R code and data necessary to reproduce our analysis has been
archived on Figshare (\emph{link here after acceptance}) and released on
GitHub (\url{https://github.com/atredennick/usses_water/releases/v0.1}).
We also include annotated Stan code in our model description in Appendix
2.

\section{RESULTS}\label{results}

Three of our five treatment years fell in years of below average
rainfall (Fig. 2A). Thus, those three years represent a lower magnitude
of absolute change in precipitation experienced by the treatments.
Averaged across treatments, ANPP varied from a minimum of 74.5 g
m\(^{-2}\) in 2014 to a maximum of 237.1 g m\(^{-2}\) in 2016 (Fig. 2C).
ANPP was slightly higher in irrigation plots and slightly lower in
drought plots (Fig. 2C), corresponding to estimated soil volumetric
water content (VWC) differences among treatments (Fig. 2B). Such
differences in soil VWC indicate our treatment infrastructure was
successful.

Cumulative growing season soil moisture had a positive effect on ANPP
(mean of \(\beta_{1}\) = 0.43; 80\% BCI = -0.05, 0.88; 95\% BCI = -0.36,
1.15) (Fig. 2D). Ecosystem functional response was similar among
treatments, with treatment level regressions between soil moisture and
log(ANPP) having similar intercepts (Fig. 3A) and slopes (Fig. 3B).
There was also no evidence that the treatment effects became more
important over time because there was no directional trend in the random
year effects (Fig. 4).

Community composition was similar among treatments. In no year did
community composition among treatments not overlap, and they were
equally dispersed in all years (Table 1; Fig. 5). Likewise, community
composition was remarkably stable over time, with no evidence of
divergence among treatments (Table 1; Fig. 5).

\section{DISCUSSION}\label{discussion}

Ecosystem response to chronic resource alteration is expected to follow
a temporal trend. Initially, ecosystem response will be modest and
reflect the physiological responses of constituent species. Over longer
time periods, species reordering will cause greater responses as species
better able to take advantage of, or cope with, new resource levels
become more abundant. This temporal trend is formalized by the
`Hierarchical Response Framework' \citep[HRF,][]{Smith2009}, and has
been empirically supported \citep{Knapp2012, Wilcox2016}. A lingering
question is not whether the HRF is a reasonable model of ecosystem
dynamics, but rather how the time scales of ecosystem response and
community change differ among ecosystems. Indeed, previous work has
shown that community compositional shifts can be both rapid, on the
order of years \citep{Hoover2014}, and slow, on order of decades
\citep{Knapp2012, Wilcox2016}. To add to this growing body of knowledge,
we performed a precipitation manipulation experiment in a sagebrush
steppe ecosystem.

Ecosystem functional response (i.e., the relationship between soil
moisture and ANPP) and community composition in this sagebrush steppe
ecosystem was consistent across drought, control, and irrigation
treatments (Fig. 2D, Fig. 3). In other words, functional responses
estimated from any precipitation level sampled apply to any other level
of precipitation, and the consistent functional response was not due to
species reordering and compensation. Thus, our results conform to the
first of our four predictions: neither ecosystem functional response nor
community composition changed with chronic alteration in water
availability (Fig. 1, top left).

This is surprising because grasslands generally, and sagebrush steppe
specifically, are considered water-limited systems. Thus, our naive
expecatations were that either (i) ecosystem functional response would
differ among treatments as species reach physiological thresholds and
resource limitations (Fig. 1, top right), or (ii) ecosystem functional
response stays the same due to species reordering and compensation (Fig.
1, bottom left).

In the absence of community composition data, an obvious hypothesis to
explain the consistent ecosystem functional response is that species
reordered rapidly. In so doing, newly abundant species could compensate
for the loss of ecosystem functioning of previously abundant species. We
did not find this to be the case, as species composition was remarkably
stable over the course of the experiment (Fig. 5).

In combination, the lack of ecosystem and community response to drought
or irrigation shows that this sagebrush steppe ecosystem is resistant to
chronic alterations in water availability. There are three possible
explanations for the resistance we found. First, it could be that our
experiment simply was not long enough to induce responses. This may be
true for community responses, which can take decades \citep{Wilcox2016}.

Second, it could be that our manipulations were not large enough to
induce a response. That is, maybe a 50\% decrease/increase in any given
year is not abnormal give our sites historical range of variability
\citep{Knapp2017}. We cannot definitively rule out this possibility, but
we have reason to believe our manipulations \emph{should} have been
large enough. Using the methods described by \citet{Lemoine2016}, we
calculated the percent reduction and increase of mean growing season
precipitation neccesary to reach the 1\% and 99\% extremes of the
historical precipitation regime at our site. The 1\% quantile of
precipitation at our site is 110 mm, a 47\% reduction from the mean, and
the 99\% quantile is 414 mm, a 77\% increase from mean growing season
precipitation (Appendix 3). Thus, our drought treatment represented
extreme precipitation amounts, especially in years where ambient
precipitation was below average (Fig. 2A). The irrigation treatment may
have been too small, however.

Third, the ecosystem and community resistance to drought and deluge we
observed could be a real phenomenon in this system, at least over the
five year time span we observed. The perennial plants in this cold
desert ecosystem are tolerant to drought conditions \citep[A.R.
Kleinhesselink, unpublished data]{Bazzaz1979, Franks2011}, meaning a
consistent ecosystem functional response at low soil moisture is
perfectly reasonable. For example, many of the perennial grasses in our
focal ecosystem avoid drought stress by growing early in the growing
season (A.R. Kleinhesselink, personal observation). Likewise, the
dominant shrub in our focal ecosystem, \emph{Artemisia tripartita}, has
access to water deep in the soil profile thanks to a deep root system
\citep{Germino2014}. As for the irrigation treatments, the lack of a
saturating response (e.g., a lower regression slope) suggests water
availability remained a limiting factor at the levels of precipitation
we were able to manipulate.

In conclusion, our results suggest that five years of \(\pm\) 50\%
ambient precipitation is not enough to induce a shift in ecosystem
functional response in a sagebrush steppe. This is likely because
component species are tolerant to drought conditions and water remains
co-limiting at higher precipitation levels. Longer time series of
chronic precipitation alteration may reveal plant community shifts that
we did not observe \citep[e.g.,][]{Wilcox2016}. Our results suggest
compositional shifts would have the largest impact at high rainfall
because the current community maintained consistent ecosystem functional
response at very low water availability.

\section{ACKNOWLEDGEMENTS}\label{acknowledgements}

We gratefully acknowledge the support of the Utah Agricultural
Experiment Station (journal paper xxxx). We thank the many summer
research technicians who collected the data reported in this paper and
the US Experimental Sheep Station for facilitating work on their
property. We also thank Susan Durham for clarifying our thinking on the
statistical analyses and Kevin Wilcox for helpful discussions on
analyzing community composition data.

\section{FUNDING}\label{funding}

NSF DBI-1400370 to Andrew Tredennick.\\
NSF Graduate Research Fellowship to Andrew Kleinhesselink.\\
NSF DEB-1353078 and DEB-1054040 to Peter Adler.

\newpage{}

\section{TABLES}\label{tables}

\begin{table}[ht]
\centering
\caption{Results from statistical tests for clustering and dispersion of community composition among precipitation treatments. `adonis` tests whether treatments form unique clusters in multidimensial space; `betadisper` tests whether treatments have similar dispersion. For both tests, \emph{P} values greater than 0.05 indicate there is no support that the treatments differ.} 
\begingroup\normalsize
\begin{tabular}{rlrrrr}
  \hline
Year & Test & n & d.f. & \emph{F} & \emph{P} \\ 
  \hline
2011 & adonis &  21 &   2 & 1.02 & 0.45 \\ 
  2011 & betadisper &  21 &   2 & 2.23 & 0.14 \\ 
  2012 & adonis &  22 &   2 & 1.10 & 0.29 \\ 
  2012 & betadisper &  22 &   2 & 0.21 & 0.81 \\ 
  2013 & adonis &  22 &   2 & 1.23 & 0.14 \\ 
  2013 & betadisper &  22 &   2 & 0.28 & 0.76 \\ 
  2014 & adonis &  22 &   2 & 0.95 & 0.58 \\ 
  2014 & betadisper &  22 &   2 & 0.35 & 0.71 \\ 
  2015 & adonis &  21 &   2 & 1.05 & 0.37 \\ 
  2015 & betadisper &  21 &   2 & 3.01 & 0.07 \\ 
  2016 & adonis &  21 &   2 & 1.07 & 0.35 \\ 
  2016 & betadisper &  21 &   2 & 0.50 & 0.62 \\ 
   \hline
\end{tabular}
\endgroup
\end{table}

\newpage{}

\section{FIGURES}\label{figures}

\begin{figure}[!ht]
  \centering
      \includegraphics[width=4in]{../figures/hypothesis_figtable.png}
  \caption{Possible outcomes of chronic resource alteration.}
\end{figure}

\newpage{}

\begin{figure}[!ht]
  \centering
      \includegraphics[height=5in]{../figures/Figure1.png}
  \caption{(A) Probability density of historical precipitation from 1926-2016, with the years of the experiment shown with arrows on the \emph{x}-axis. (B) Observed soil volumetric water content (VWC) over the course of the experiment. (C) Mean (points) ANPP and its standard deviation (error bars) for each year of the experiment. (D) Scatterplot of ANPP versus growing season precipitation. Colored regression lines are independently-fit linear models for each treatment with no random effects structure; dark black regression line is a linear regression through all the point with no random effects structure. Our analysis seeks to find if the data supports a model with different intercepts and/or slopes for each treatment, while also accounting for nonindependence of samples within plots across years. Color mapping for panels B-D is shown in the legend for panel C.}
\end{figure}

\newpage{}

\begin{figure}[!ht]
  \centering
      \includegraphics[width=5in]{../figures/glmm_treatment_posteriors.png}
  \caption{Posterior distributions of treatment-level parameters ('Intercept' and the effect of 'Soil Moisture'). Kernel bandwidths of posterior densities were adjusted by a factor of 5 for visual clarity.}
\end{figure}

\newpage{}

\begin{figure}[!ht]
  \centering
      \includegraphics[width=5in]{../figures/glmm_yeardiffs.png}
  \caption{Posterior distributions of random year effects (intercept offsets). Kernel bandwidths of posterior densities were adjusted by a factor of 5 for visual clarity.}
\end{figure}

\newpage{}

\begin{figure}[!ht]
  \centering
      \includegraphics[width=5in]{../figures/sppcomp_bray_all.png}
  \caption{Nonmetric multidimensional scaling scores representing plant communities in each plot, colored by treatment.}
\end{figure}

\newpage{}


\bibliography{/Users/atredenn/Dropbox/Bibliography/usses_water.bib}


\end{document}
